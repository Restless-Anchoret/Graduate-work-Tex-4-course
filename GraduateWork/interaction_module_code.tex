\section*{\texttt{Main.java}}
\begin{verbatim}
package com.ran.interaction.main;

// Импорт классов

public class Main {

    private static void plugInClasses() {
        JavaClassChecker.plugInClass();
    }

    public static void main(String[] args) {
        plugInClasses();
        MainController controller = new MainController();
        controller.init();
        controller.showFrame();
    }

}
\end{verbatim}

\section*{\texttt{Observer.java}}
\begin{verbatim}
package com.ran.interaction.observer;

@FunctionalInterface
public interface Observer {

    void notify(String id, Object parameter);
    
}
\end{verbatim}

\section*{\texttt{Publisher.java}}
\begin{verbatim}
package com.ran.interaction.observer;

import java.util.List;

public interface Publisher {

    List<String> getAvailableIds();
    void subscribe(String id, Observer observer);
    Observer getObserver(String id);
    
    default void initObservers() {
        for (String id: getAvailableIds()) {
            subscribe(id, EmptyObserver.getInstanse());
        }
    }
    
    default void notifyObservers(Object parameter) {
        for (String id: getAvailableIds()) {
            getObserver(id).notify(id, parameter);
        }
    }
    
}
\end{verbatim}

\section*{\texttt{MainFrame.java}}
\begin{verbatim}
package com.ran.interaction.windows;

// Импорт классов

public class MainFrame extends JFrame {

    private static final String SUBMISSIONS = "Submissions";
    private static final String PROBLEMS = "Problems";
    
    public MainFrame() {
        setDefaultCloseOperation(JFrame.EXIT_ON_CLOSE);
        initComponents();
        initCustomComponents();
    }

    // Код создания интерфейса окна, поля с компонентами                  

    private SubmissionsPanel submissionsPanel;
    private ProblemsPanel problemsPanel;

    private void initCustomComponents() {
        submissionsPanel = new SubmissionsPanel();
        tabbedPane.add(submissionsPanel);
        tabbedPane.setTitleAt(0, SUBMISSIONS);
        problemsPanel = new ProblemsPanel();
        tabbedPane.add(problemsPanel);
        tabbedPane.setTitleAt(1, PROBLEMS);
    }
    
    public SubmissionsPanel getSubmissionsPanel() {
        return submissionsPanel;
    }

    public ProblemsPanel getProblemsPanel() {
        return problemsPanel;
    }
    
}
\end{verbatim}

\section*{\texttt{SubmissionsPanel.java}}
\begin{verbatim}
package com.ran.interaction.panels;

// Импорт классов

public class SubmissionsPanel extends JPanel implements Publisher {

    public static final String ADD = "add_submission";
    public static final String RESUBMIT = "resubmit_submission";
    public static final String DELETE = "delete_submission";
    public static final String UPDATE = "update_submissions";
    public static final String VIEW_CODE = "view_submission_code";
    
    // Дополнительные статические константы
    
    public SubmissionsPanel() {
        initComponents();
        initCustomComponents();
        setTableContent(new Object[][] { });
        initObservers();
    }

    // Код создания интерфейса окна, поля с компонентами                     

    private Map<String, Observer> observers = new HashMap<>();
    
    private void initCustomComponents() {
        buttonAdd.addActionListener(event -> getObserver(ADD).notify(ADD, null));
        buttonResubmit.addActionListener(event -> callObserverIfRowIsSelected(RESUBMIT));
        buttonDelete.addActionListener(event -> callObserverIfRowIsSelected(DELETE));
        buttonViewCode.addActionListener(event -> callObserverIfRowIsSelected(VIEW_CODE));
        buttonUpdate.addActionListener(event -> getObserver(UPDATE).notify(UPDATE, null));
    }
    
    private void callObserverIfRowIsSelected(String id) {
        if (tableSubmissions.getSelectedIdentifier() != null) {
            getObserver(id).notify(id, tableSubmissions.getSelectedIdentifier());
        }
    }
    
    @Override
    public List<String> getAvailableIds() {
        return Arrays.asList(ADD, RESUBMIT, DELETE, VIEW_CODE, UPDATE);
    }
    
    @Override
    public void subscribe(String id, Observer observer) {
        observers.put(id, observer);
    }

    @Override
    public Observer getObserver(String id) {
        return observers.getOrDefault(id, EmptyObserver.getInstanse());
    }
    
    public final void setTableContent(Object[][] content) {
        tableSubmissions.setTableContent(content, TABLE_HEADERS);
    }
    
}
\end{verbatim}

\section*{\texttt{ProblemsPanel.java}}
\begin{verbatim}
package com.ran.interaction.panels;

// Импорт классов

public class ProblemsPanel extends JPanel implements Publisher {
    
    public static final String ADD = "add_problem";
    public static final String EDIT = "edit_problem";
    public static final String DELETE = "delete_problem";
    public static final String UPDATE = "update_problems";
    
    // Дополнительные статические константы
    
    public ProblemsPanel() {
        initComponents();
        initCustomComponents();
        setTableContent(new Object[][] { });
        initObservers();
    }

    // Код создания интерфейса окна, поля с компонентами

    private Map<String, Observer> observers = new HashMap<>();
    
    private void initCustomComponents() {
        buttonAdd.addActionListener(event -> getObserver(ADD).notify(ADD, null));
        buttonEdit.addActionListener(event -> callObserverIfRowIsSelected(EDIT));
        buttonDelete.addActionListener(event -> callObserverIfRowIsSelected(DELETE));
        buttonUpdate.addActionListener(event -> getObserver(UPDATE).notify(UPDATE, null));
    }
    
    private void callObserverIfRowIsSelected(String id) {
        if (tableProblems.getSelectedIdentifier() != null) {
            getObserver(id).notify(id, tableProblems.getSelectedIdentifier());
        }
    }
    
    @Override
    public List<String> getAvailableIds() {
        return Arrays.asList(ADD, EDIT, DELETE, UPDATE);
    }
    
    @Override
    public void subscribe(String id, Observer observer) {
        observers.put(id, observer);
    }

    @Override
    public Observer getObserver(String id) {
        return observers.getOrDefault(id, EmptyObserver.getInstanse());
    }
    
    public final void setTableContent(Object[][] content) {
        tableProblems.setTableContent(content, TABLE_HEADERS);
    }
    
}
\end{verbatim}

\section*{\texttt{MainController.java}}
\begin{verbatim}
package com.ran.interaction.controllers;

// Импорт классов

public class MainController {

    private ProblemsCreator creator;
    private MainFrame mainFrame;

    public void init() {
        creator = new ProblemsCreator();
        creator.init();
    }
    
    public void showFrame() {
        EventQueue.invokeLater(() -> {
            try {
                for (UIManager.LookAndFeelInfo info : UIManager.getInstalledLookAndFeels()) {
                    if ("Nimbus".equals(info.getName())) {
                        UIManager.setLookAndFeel(info.getClassName());
                        break;
                    }
                }
            } catch (ClassNotFoundException | InstantiationException |
                    IllegalAccessException | UnsupportedLookAndFeelException exception) {
                InteractionLogging.logger.log(Level.FINE,
                        "Cannot set Nimbus Look and Feel", exception);
            }
            mainFrame = new MainFrame();
            configurateMainFrame(mainFrame);
            mainFrame.setVisible(true);
        });
    }

    private void configurateMainFrame(MainFrame mainFrame) {
        mainFrame.addWindowListener(new WindowAdapter() {
            public void windowClosing(WindowEvent event) {
                creator.stop();
            }
        });
        SubmissionsPanel submissionsPanel = mainFrame.getSubmissionsPanel();
        updateSubmissions(null, null);
        submissionsPanel.subscribe(SubmissionsPanel.ADD, this::addSubmission);
        submissionsPanel.subscribe(SubmissionsPanel.RESUBMIT, this::submitSubmission);
        submissionsPanel.subscribe(SubmissionsPanel.VIEW_CODE, this::viewSubmissionCode);
        submissionsPanel.subscribe(SubmissionsPanel.DELETE, this::deleteSubmission);
        submissionsPanel.subscribe(SubmissionsPanel.UPDATE, this::updateSubmissions);
        ProblemsPanel problemsPanel = mainFrame.getProblemsPanel();
        updateProblems(null, null);
        problemsPanel.subscribe(ProblemsPanel.ADD, this::addProblem);
        problemsPanel.subscribe(ProblemsPanel.EDIT, this::editProblem);
        problemsPanel.subscribe(ProblemsPanel.DELETE, this::deleteProblem);
        problemsPanel.subscribe(ProblemsPanel.UPDATE, this::updateProblems);
    }
    
    private void addSubmission(String id, Object parameter) {
        SubmissionCreationController creationController = new SubmissionCreationController();
        creationController.setFileSupplier(creator.getFileSupplier());
        creationController.showDialog();
        String submissionFolder = creationController.getSubmissionFolder();
        if (submissionFolder != null) {
            updateSubmissions(null, null);
            submitSubmission(null, submissionFolder);
        }
    }
    
    private void submitSubmission(String id, Object parameter) {
        String submissionFolder = parameter.toString();
        SubmissionResultController resultController = new SubmissionResultController();
        resultController.setFileSupplier(creator.getFileSupplier());
        resultController.setTestingSystem(creator.getTestingSystem());
        resultController.setSubmissionFolder(submissionFolder);
        resultController.showDialog();
        updateSubmissions(null, null);
    }
    
    private void viewSubmissionCode(String id, Object parameter) {
        String submissionFolder = parameter.toString();
        Path sourceFilePath = creator.getFileSupplier()
                .getSubmissionCodeSupplier(submissionFolder).getSourceFile();
        FileEditorController editorController = new FileEditorController();
        editorController.showDialog(sourceFilePath, true);
    }
    
    private void deleteSubmission(String id, Object parameter) {
        int answer = SwingUtil.showYesNoDialog(mainFrame,
                SubmissionsPanel.DELETING_MESSAGE, SubmissionsPanel.DELETING_TITLE);
        if (answer == JOptionPane.YES_OPTION) {
            creator.getFileSupplier().deleteSubmissionFolder(parameter.toString());
            updateSubmissions(null, null);
        }
    }
    
    private void updateSubmissions(String id, Object parameter) {
        FileSupplier fileSupplier = creator.getFileSupplier();
        Properties properties = PresentationSupport.getPresentationProperties();
        List<String> submissionNumbers = fileSupplier.getSubmissionsFolderNames();
        Object[][] content = SwingUtil.prepareTableContent(submissionNumbers, (number, row) -> {
            row.add(number);
            SubmissionDescriptor descriptor = fileSupplier.getSubmissionDescriptor(number);
            String problemNumber = descriptor.getProblemName();
            row.add(fileSupplier.getProblemsFolderNames().contains(problemNumber) ?
                    fileSupplier.getProblemDescriptor(problemNumber).getProblemName() : "");
            row.add(properties.getProperty(descriptor.getEvaluationType()));
            row.add(properties.getProperty(descriptor.getCompilatorName()));
            row.add(TestingUtil.getVerdictDescription(descriptor.getVerdict(),
                    descriptor.getDecisionPoints(), descriptor.getWrongTestNumber()));
            row.add(TestingUtil.getTimeDescription(descriptor.getDecisionTime()));
        });
        mainFrame.getSubmissionsPanel().setTableContent(content);
    }
    
    private void addProblem(String id, Object parameter) {
        creator.getFileSupplier().addProblemFolder();
        updateProblems(null, null);
    }
    
    private void editProblem(String id, Object parameter) {
        ProblemController problemController = new ProblemController();
        problemController.setFileSupplier(creator.getFileSupplier());
        problemController.setTestingSystem(creator.getTestingSystem());
        problemController.setProblemFolder(parameter.toString());
        problemController.showDialog();
        updateProblems(null, null);
        updateSubmissions(null, null);
    }
    
    private void deleteProblem(String id, Object parameter) {
        int answer = SwingUtil.showYesNoDialog(mainFrame,
                ProblemsPanel.DELETING_MESSAGE, ProblemsPanel.DELETING_TITLE);
        if (answer == JOptionPane.YES_OPTION) {
            creator.getFileSupplier().deleteProblemFolder(parameter.toString());
            updateProblems(null, null);
        }
    }
    
    private void updateProblems(String id, Object parameter) {
        FileSupplier fileSupplier = creator.getFileSupplier();
        List<String> problemNumbers = fileSupplier.getProblemsFolderNames();
        Properties presentationProperties = PresentationSupport.getPresentationProperties();
        Object[][] content = SwingUtil.prepareTableContent(problemNumbers, (number, row) -> {
            row.add(number);
            ProblemDescriptor descriptor = fileSupplier.getProblemDescriptor(number);
            row.add(descriptor.getProblemName());
            row.add(TestingUtil.getTimeDescription(descriptor.getTimeLimit()));
            row.add(TestingUtil.getMemoryDescription(descriptor.getMemoryLimit()));
            row.add(presentationProperties.getProperty(descriptor.getCheckerType()));
        });
        mainFrame.getProblemsPanel().setTableContent(content);
    }

}
\end{verbatim}

\section*{\texttt{ProblemsCreator.java}}
\begin{verbatim}
package com.ran.interaction.controllers;

// Импорт классов

public class ProblemsCreator {

    private static final String TESTING_SYSTEM_THREADS_DEFAULT = "10";
    private static final String FILE_SYSTEM_PATH_DEFAULT = System.getProperty("user.dir");

    private TestingSystem testingSystem;
    private FileSupplier fileSupplier;

    public void init() {
        Map<String, String> creatorParams = new CreatorParamsReader().getCreatorParams();
        String fileSystemPath = creatorParams.getOrDefault(
                CreatorParamsReader.FILE_SYSTEM_PATH_PARAM_NAME, FILE_SYSTEM_PATH_DEFAULT);
        fileSupplier = new StandardFileSupplier(Paths.get(fileSystemPath));
        String testingSystemThreads = creatorParams.getOrDefault(CreatorParamsReader
                .TESTING_SYSTEM_THREADS_PARAM_NAME, TESTING_SYSTEM_THREADS_DEFAULT);
        MultithreadTestingSystem multithreadTestingSystem
                = MultithreadTestingSystem.getDefault();
        multithreadTestingSystem.setThreadsQuantity(Integer.parseInt(testingSystemThreads));
        multithreadTestingSystem.setFileSupplier(new TestingFileSupplierImpl(fileSupplier));
        testingSystem = multithreadTestingSystem;
        testingSystem.start();
    }

    public void stop() {
        testingSystem.stop();
    }

    public TestingSystem getTestingSystem() {
        return testingSystem;
    }

    public FileSupplier getFileSupplier() {
        return fileSupplier;
    }

}

class CreatorParamsReader {

    public static final String CREATOR_PARAMS_FILE = "creator-params.xml";
    public static final String TESTING_SYSTEM_THREADS_PARAM_NAME = "testingSystemThreads";
    public static final String FILE_SYSTEM_PATH_PARAM_NAME = "fileSystemPath";
    public static final String PARAM_NAME_NODE_NAME = "name";
    public static final String PARAM_VALUE_NODE_NAME = "value";

    public Map<String, String> getCreatorParams() {
        Map<String, String> creatorParams = new HashMap<>();
        try (InputStream creatorParamStream = ProblemsCreator.class
                .getResourceAsStream(CREATOR_PARAMS_FILE)) {
            DocumentBuilder documentBuilder = DocumentBuilderFactory.newInstance()
                    .newDocumentBuilder();
            Document document = documentBuilder.parse(creatorParamStream);
            Element rootElement = document.getDocumentElement();
            NodeList paramList = rootElement.getChildNodes();
            for (int i = 0; i < paramList.getLength(); i++) {
                Node node = paramList.item(i);
                if (node instanceof Element) {
                    updateCreatorParamsMap(creatorParams, node);
                }
            }
        } catch (Exception exception) {
            InteractionLogging.logger.log(Level.FINE,
                    "Exception while reading creator-params.xml", exception);
        }
        return creatorParams;
    }

    private void updateCreatorParamsMap(Map<String, String> creatorParams, Node paramNode) {
        NodeList nodeList = paramNode.getChildNodes();
        String paramName = null;
        String paramValue = null;
        for (int i = 0; i < nodeList.getLength(); i++) {
            Node node = nodeList.item(i);
            if (node instanceof Element) {
                String nodeName = node.getNodeName();
                String nodeValue = node.getFirstChild().getNodeValue();
                switch (nodeName) {
                    case PARAM_NAME_NODE_NAME: paramName = nodeValue; break;
                    case PARAM_VALUE_NODE_NAME: paramValue = nodeValue; break;
                }
            }
        }
        creatorParams.put(paramName, paramValue);
    }
}
\end{verbatim}
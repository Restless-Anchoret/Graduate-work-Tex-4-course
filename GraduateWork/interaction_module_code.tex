\section*{\texttt{Main.java}}
\begin{verbatim}
package com.ran.interaction.main;

// Импорт классов
// ...

public class Main {

    // Подключение класса специального чекера
    private static void plugInClasses() {
        JavaClassChecker.plugInClass();
    }

    // Запуск главного контроллера
    public static void main(String[] args) {
        plugInClasses();
        MainController controller = new MainController();
        controller.init();
        controller.showFrame();
    }

}
\end{verbatim}

\section*{\texttt{Observer.java}}
\begin{verbatim}
package com.ran.interaction.observer;

// Интерфейс наблюдателя
@FunctionalInterface
public interface Observer {

    // Метод для оповещения наблюдателя
    void notify(String id, Object parameter);
    
}
\end{verbatim}

\section*{\texttt{Publisher.java}}
\begin{verbatim}
package com.ran.interaction.observer;

import java.util.List;

// Интерфейс издателя
public interface Publisher {

    // Метод доступа к списку доступных идентификаторов событий
    List<String> getAvailableIds();
    // Метод подписки наблюдателя к определённому событию
    void subscribe(String id, Observer observer);
    Observer getObserver(String id);
    
    // Метод инициализации наблюдателей по умолчанию
    default void initObservers() {
        for (String id: getAvailableIds()) {
            subscribe(id, EmptyObserver.getInstanse());
        }
    }
    
    // Метод оповещения наблюдателей об определённом событии
    default void notifyObservers(Object parameter) {
        for (String id: getAvailableIds()) {
            getObserver(id).notify(id, parameter);
        }
    }
    
}
\end{verbatim}

\section*{\texttt{SubmissionsPanel.java}}
\begin{verbatim}
package com.ran.interaction.panels;

// Импорт классов
// ...

// Панель с посылками
public class SubmissionsPanel extends JPanel implements Publisher {

    // Идентификаторы событий панели
    public static final String ADD = "add_submission";
    public static final String RESUBMIT = "resubmit_submission";
    public static final String DELETE = "delete_submission";
    public static final String UPDATE = "update_submissions";
    public static final String VIEW_CODE = "view_submission_code";
    
    // Дополнительные статические константы
    // ...
    
    public SubmissionsPanel() {
        initComponents();
        initCustomComponents();
        setTableContent(new Object[][] { });
        initObservers();
    }

    // Код создания интерфейса окна, поля с компонентами, вспомогательные методы
    // ...                    

    // Набор зарегистрированных наблюдателей
    private Map<String, Observer> observers = new HashMap<>();
    
    // Настройка механизма оповещения наблюдателей о событиях
    private void initCustomComponents() {
        buttonAdd.addActionListener(event -> getObserver(ADD).notify(ADD, null));
        buttonResubmit.addActionListener(event -> callObserverIfRowIsSelected(RESUBMIT));
        buttonDelete.addActionListener(event -> callObserverIfRowIsSelected(DELETE));
        buttonViewCode.addActionListener(event -> callObserverIfRowIsSelected(VIEW_CODE));
        buttonUpdate.addActionListener(event -> getObserver(UPDATE).notify(UPDATE, null));
    }
    
    private void callObserverIfRowIsSelected(String id) {
        if (tableSubmissions.getSelectedIdentifier() != null) {
            getObserver(id).notify(id, tableSubmissions.getSelectedIdentifier());
        }
    }
    
    @Override
    public List<String> getAvailableIds() {
        return Arrays.asList(ADD, RESUBMIT, DELETE, VIEW_CODE, UPDATE);
    }
    
    @Override
    public void subscribe(String id, Observer observer) {
        observers.put(id, observer);
    }

    @Override
    public Observer getObserver(String id) {
        return observers.getOrDefault(id, EmptyObserver.getInstanse());
    }
    
}
\end{verbatim}

\section*{\texttt{MainController.java}}
\begin{verbatim}
package com.ran.interaction.controllers;

// Импорт классов
// ...

// Главный контроллер приложения
public class MainController {

    // Объединённые ядро тестирующей системы и поставщик файлов
    private ProblemsCreator creator;
    // Главное окно приложения
    private MainFrame mainFrame;

    public void init() {
        creator = new ProblemsCreator();
        creator.init();
    }
    
    public void showFrame() {
        EventQueue.invokeLater(() -> {
            try {
                for (UIManager.LookAndFeelInfo info : UIManager.getInstalledLookAndFeels()) {
                    if ("Nimbus".equals(info.getName())) {
                        UIManager.setLookAndFeel(info.getClassName());
                        break;
                    }
                }
            } catch (ClassNotFoundException | InstantiationException |
                    IllegalAccessException | UnsupportedLookAndFeelException exception) {
                InteractionLogging.logger.log(Level.FINE,
                        "Cannot set Nimbus Look and Feel", exception);
            }
            mainFrame = new MainFrame();
            configurateMainFrame(mainFrame);
            mainFrame.setVisible(true);
        });
    }

    // Настройка главного окна приложения
    // Подписка наблюдателей
    private void configurateMainFrame(MainFrame mainFrame) {
        mainFrame.addWindowListener(new WindowAdapter() {
            public void windowClosing(WindowEvent event) {
                creator.stop();
            }
        });
        SubmissionsPanel submissionsPanel = mainFrame.getSubmissionsPanel();
        updateSubmissions(null, null);
        submissionsPanel.subscribe(SubmissionsPanel.ADD, this::addSubmission);
        submissionsPanel.subscribe(SubmissionsPanel.RESUBMIT, this::submitSubmission);
        submissionsPanel.subscribe(SubmissionsPanel.VIEW_CODE, this::viewSubmissionCode);
        submissionsPanel.subscribe(SubmissionsPanel.DELETE, this::deleteSubmission);
        submissionsPanel.subscribe(SubmissionsPanel.UPDATE, this::updateSubmissions);
        ProblemsPanel problemsPanel = mainFrame.getProblemsPanel();
        // Аналогичная подписка наблюдателей для панели problemsPanel
        // ...
    }
    
    // Реализация методов обработки событий
    // ...
}
\end{verbatim}
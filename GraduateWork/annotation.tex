Олимпиады по программированию сегодня "--- очень распространённое и актуальное явление. Они проводятся как среди школьников, так и среди студентов и профессиональных работников, как на уровне вузов, так и на международном уровне. Многие крупные IT-компании, такие как Google, Facebook, Яндекс, Mail.ru, ВКонтакте, регулярно проводят свои собственные олимпиады.

Проведение олимпиад по программированию требует наличия специального программного обеспечения: тестирующей системы, удовлетворяющей определённому набору требований, а также "--- системы для разработки задач. Подготовка наборов тестов для задач "--- весьма трудоёмкий процесс, требующий применения специальных инструментов, таких как генераторы, валидаторы, чекеры и интеракторы.

В данной работе ставится цель изучить основные аспекты работы тестирующих систем и систем разработки задач для олимпиад по программированию, а также "--- написать программу с удобным графическим пользовательским интерфейсом, предоставляющую возможность разрабатывать новые задачи и проверять решения на подготовленных тестах. Предполагается написание программы на языке Java с использованием библиотеки Swing.
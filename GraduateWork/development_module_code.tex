\section*{\texttt{Generator.java}}
\begin{verbatim}
package com.ran.development.gen;

// Импорт классов
// ...

public abstract class Generator {

    private static final long DEFAULT_RANDOM_SEED = 1938572538L;
    
    public static Generator getDefault() {
        return new Generator() {
            @Override
            protected void generate(String[] args) { }
        };
    }
    
    // Генератор псевдослучайных чисел
    private Random random;
    // Поток вывода в файл
    private PrintStream output;
    // Парсер регулярных выражений
    private RegularExpressionParser parser = new RegularExpressionParser();
    
    // Контструкторы, геттеры, сеттеры
    // ...
    
    // print-методы
    // ...
    
    protected int nextInt(int higherBound) {
        return random.nextInt(higherBound);
    }
    
    protected int nextInt(int lowerBound, int higherBound) {
        return lowerBound + random.nextInt(higherBound - lowerBound + 1);
    }
    
    // Ещё next-методы
    // ...
    
    protected int any(int[] values) {
        return values[random.nextInt(values.length)];
    }
    
    protected char any(String regularExpression) {
        int quantity = parser.getCharactersQuantity(regularExpression);
        return parser.getCharacter(regularExpression, random.nextInt(quantity));
    }
    
    // Ещё any-методы
    // ...
    
    // Метод непосредственной генерации входных данных
    abstract protected void generate(String[] args);
    
}
\end{verbatim}

\section*{\texttt{MultiGenerator.java}}
\begin{verbatim}
package com.ran.development.gen;

// Импорт классов
// ...

public class MultiGenerator {
    
    private static final long MESSAGING_DELAY = 2000;
    
    // Набор слушателей процесса генерации
    private DevelopmentListenerSupport listenerSupport = new DevelopmentListenerSupport();
    // Поставщик генератора
    private Supplier<? extends Generator> generatorSupplier = () -> {
        return Generator.getDefault();
    };
    // Инициализирующее значение генератора псевдослучайных чисел
    private int randomSeed = 0;
    // Пути файлов для генерации входных данных
    private Path[] paths = { };
    // Аргументы запуска генератора
    private String[] arguments = { };
    
    // Геттеры, сеттеры, методы работы со слушателями
    // ...
    
    // Запуск генератора в отдельном потоке
    public void performGenerating() {
        listenerSupport.fireProcessingStarted();
        Generator generator = generatorSupplier.get();
        if (generator == null) {
            finishEarlier("Cannot instantiate Generator subclass",
                    DevelopmentResult.FAIL, 1, paths.length);
            return;
        }
        generator.setRandomSeed(randomSeed);
        for (int index = 1; index <= paths.length; index++) {
            listenerSupport.fireTaskProcessingStarted(index);
            Path path = paths[index - 1];
            long start = System.currentTimeMillis();
            Thread thread = null;
            try (OutputStream output = Files.newOutputStream(path)) {
                generator.setOutput(output);
                FutureTask<DevelopmentResult> futureTask = new FutureTask<>(
                        new GeneratingTask(generator, index, arguments));
                thread = new Thread(futureTask);
                start = System.currentTimeMillis();
                thread.start();
                while (thread.isAlive()) {
                    thread.join(MESSAGING_DELAY);
                    if (thread.isAlive()) {
                        listenerSupport.fireTaskIsProcessing(index,
                                System.currentTimeMillis() - start);
                    }
                }
                listenerSupport.fireTaskIsDone(futureTask.get());
            } // Перехват исключений, вывод в лог
            // ...
        }
        listenerSupport.fireProcessingFinished();
    }
    
    // Вспомогательные методы
    // ...
    
    // Задача запуска генератора
    private static class GeneratingTask implements Callable<DevelopmentResult> {
        private String[] arguments;
        private Generator generator;
        private int generatorNumber;
        
        // Конструктор
        // ...
        
        @Override
        public DevelopmentResult call() {
            long start = System.currentTimeMillis();
            try {
                generator.generate(arguments);
            } // Перехват исключений, вывод в лог
            // ...
            return new DevelopmentResult(generatorNumber, System.currentTimeMillis() - start);
        }
    }
    
}
\end{verbatim}

\section*{\texttt{Validator.java}}
\begin{verbatim}
package com.ran.development.valid;

// Импорт классов
// ...

public abstract class Validator {

    public static Validator getDefault() {
        return new Validator() {
            @Override
            public void validate(String[] args) throws ValidationException { }
        };
    }
    
    // Строгий ввод валидируемых входных данных
    private StrictInput input = StrictInput.getEmptyInput();
    // Парсер регулярных выражений
    private RegularExpressionParser parser = new RegularExpressionParser();

    // Конструкторы, геттеры, сеттеры
    // ...
    
    protected void error() throws ValidationException {
        throw new ValidationException(input.getState());
    }
    
    protected void error(String message) throws ValidationException {
        throw new ValidationException(message, input.getState());
    }
    
    protected void ensure(boolean condition) throws ValidationException {
        if (!condition) {
            error("Condition is not true");
        }
    }
    
    protected void inBounds(int lowEdge, int number, int highEdge)
            throws ValidationException {
        if (!(lowEdge <= number && number <= highEdge)) {
            error("Number out of bounds");
        }
    }
    
    // Ещё inBounds-методы
    // ...
    
    protected void matchesExpression(char symbol, String regularExpression)
            throws ValidationException {
        if (!parser.matchesExpression(symbol, regularExpression)) {
            error("Symbol does not match the regular expression");
        }
    }
    
    // Ещё matchesExpression-методы
    // ...
    
    protected void inValues(int number, int[] values) throws ValidationException {
        for (int value: values) {
            if (value == number) {
                return;
            }
        }
        error("Number is not in values array");
    }
    
    // Ещё inValues-методы
    // ...
    
    // Метод-обёртка для проверки, что файл прочитал полностью
    public void performValidation(String[] args) throws ValidationException {
        validate(args);
        if (!input.isReaden()) {
            error("File was not readen completely");
        }
    }
    
    // Метод непосредственной валидации входных данных
    public abstract void validate(String[] args) throws ValidationException;
    
}
\end{verbatim}

\section*{\texttt{Checker.java}}
\begin{verbatim}
package com.ran.development.check;

// Импорт классов
// ...

public abstract class Checker {

    public static final int OK = 0,
            WRONG_ANSWER = 1,
            FAIL = 2;

    // Лексемный ввод входных данных
    private TokenInput input = TokenInput.getEmptyInput();
    // Лексемный ввод ответа участника
    private TokenInput output = TokenInput.getEmptyInput();
    // Лексемный ввод ответа жюри
    private TokenInput answer = TokenInput.getEmptyInput();
    // Парсер регулярных выражений
    private RegularExpressionParser parser = new RegularExpressionParser();

    // Геттеры, сеттеры
    // ...

    protected void quit(int resultInfo, String message) throws CheckResultException {
        throw new CheckResultException(resultInfo, message);
    }

    // Ещё quit-методы
    // ...

    protected void quitIf(boolean condition, int resultInfo, String message)
            throws CheckResultException {
        if (condition) {
            quit(resultInfo, message);
        }
    }

    // Ещё quitIf-методы
    // ...

    protected void ensureIf(boolean condition, String message) throws CheckResultException {
        if (!condition) {
            quit(WRONG_ANSWER, message);
        }
    }

    // Ещё ensureIf-методы
    // ...

    // Метод-обёртка для перехвата исключения
    public void performChecking() throws CheckResultException {
        try {
            check();
        } catch (CheckResultException exception) {
            if (exception.getResultInfo() == OK && !output.checkEof()) {
                quit(WRONG_ANSWER, "Extra values in the output");
            }
            throw exception;
        }
        quit(FAIL, "Checker did not tell result of checking");
    }

    // Метод непосредственной проверки ответа
    public abstract void check() throws CheckResultException;

}
\end{verbatim}

\section*{\texttt{Utils.java}}
\begin{verbatim}
package com.ran.development.util;

// Импорт классов
// ...

// Класс для динамической загрузки классов
public class Utils {

    private Utils() { }
    
    public static <T> Supplier<? extends T> getSupplier(Class<T> parentClass,
            Path classFilePath) {
        try {
            Path folderPath = classFilePath.getParent();
            String className = getClassName(classFilePath);
            ClassLoader loader = new URLClassLoader(new URL[] { folderPath.toUri().toURL() });
            Class<? extends T> cl = (Class<? extends T>)loader.loadClass(className);
            cl.newInstance();
            return () -> {
                try {
                    return cl.newInstance();
                } catch (InstantiationException | IllegalAccessException exception) {
                    String message = "Cannot create instance of " + parentClass.getName() +
                            " subclass";
                    DevelopmentLogging.logger.log(Level.FINE, message, exception);
                    return null;
                }
            };
        } // Перехват исключений, вывод в лог
        // ...
    }
    
    // Вспомогательные методы
    // ...
    
}
\end{verbatim}
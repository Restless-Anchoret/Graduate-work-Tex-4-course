В данной работе была поставлена задача написать приложение для разработки задач по олимпиадному программированию и для тестирования решений. Предполагалось использование языка программирования Java и библиотеки компонентов графического интерфейса Swing.

Для решения поставленной задачи были рассмотрены основные принципы разработки задач и тестирования решений. Были описаны правила написания кода таких средств, как генераторы, валидаторы и чекеры, рассмотрены алгоритм тестирования посылки, различные системы оценивания и используемые вердикты, присуждаемые решениям.

Была спроектирована архитектура приложения, которое включило четыре модуля: библиотеку для разработки задач, а также модули для тестирования, доступа к файловой системе и взаимодействия с пользователем. Приложение было разработано, и практически все пакеты и классы, вошедшие в него, были подробно описаны. Наконец, был приведён пример работы приложения, описывающий процесс разработки простой задачи.

Ясно, что тематика, затронутая в данной работе, весьма актуальна сегодня, поэтому разработанное приложение вполне может найти себе применение, оказав большую помощь при подготовке тестов к задачам. Кроме того, функционал данного приложения возможно расширять и улучшать.

%В ходе курсовой работы было исследовано четыре различных алгоритма поиска выравниваний пар строк, обладающих определёнными характеристиками. Мы подробно обсудили, как используется в них принцип динамического программирования и каким образом строится искомое оптимальное выравнивание, поговорили об эффективности каждого из этих алгоритмов, а также дополнительно для двух из них выявили способ нахождения количества кооптимальных выравниваний.

%Была поставлена задача написать приложение, реализующее все четыре алгоритма и позволяющее запускать их с некоторым набором входных параметров. В ходе работы это приложение было разработано, в процессе чего активно использовались объектно-ориентированное программирование и паттерны проектирования. Была подробно описана структура классов, использованных в приложении, и приведён пример его работы.

%Ясно, что в созданном приложении реализованы лишь некоторые алгоритмы, работающие с выравниваниями строковых последовательностей, и что помимо них существует ещё множество подобных алгоритмов, из чего следует возможность дальнейшей разработки данного приложения.
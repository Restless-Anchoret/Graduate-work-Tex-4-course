В данной работе была поставлена задача написать приложение для разработки задач по олимпиадному программированию и для тестирования решений. Предполагалось использование языка программирования Java и библиотеки компонентов графического интерфейса Swing.

Для решения поставленной задачи были рассмотрены основные принципы разработки задач и тестирования решений. Были описаны правила написания кода таких средств, как генераторы, валидаторы и чекеры, рассмотрены алгоритм тестирования посылки, различные системы оценивания и используемые вердикты, присуждаемые решениям.

Была спроектирована архитектура приложения, которое включило четыре модуля: библиотеку для разработки задач, а также модули для тестирования, доступа к файловой системе и взаимодействия с пользователем. Приложение было разработано, и практически все пакеты и классы, вошедшие в него, были подробно описаны. Наконец, был приведён пример работы приложения, описывающий процесс разработки простой задачи.

Ясно, что тематика, затронутая в данной работе, весьма актуальна сегодня, поэтому разработанное приложение вполне может найти себе применение, оказав большую помощь при подготовке тестов к задачам. Кроме того, функциональные возможности данного приложения возможно расширять и улучшать.
\section*{\texttt{FileSupplier.java}}
\begin{verbatim}
package com.ran.filesystem.supplier;

// Импорт классов
// ...

// Главный интерфейс поставщика файлов
public interface FileSupplier {

    // Методы для работы с папками задач
    String addProblemFolder();
    void deleteProblemFolder(String problemFolder);
    List<String> getProblemsFolderNames();
    ProblemDescriptor getProblemDescriptor(String problemFolder);
    Path getProblemStatementPath(String problemFolder);
    boolean putProblemStatementPath(String problemFolder, Path newStatementPath);
    
    // Методы для работы с папками тестов
    boolean addTestInputFiles(String problemFolder, String testGroupType,
            List<Path> inputFilePaths);
    boolean addTestAnswerFile(String problemFolder, String testGroupType, int testNumber,
            Path answerFilePath);
    void deleteTests(String problemFolder, String testGroupType, List<Integer> testNumbers);
    Path getTestInputFile(String problemFolder, String testGroupType, int testNumber);
    Path getTestAnswerFile(String problemFolder, String testGroupType, int testNumber);
    int getTestsQuantity(String problemFolder, String testGroupType);
    TestGroupDescriptor getTestGroupDescriptor(String problemFolder, String testGroupType);
    
    // Метод для работы с папкой чекера
    CodeSupplier getCheckerCodeSupplier(String problemFolder);

    // Методы для работы с папками генераторов
    String addGeneratorFolder(String problemFolder);
    void deleteGeneratorFolder(String problemFolder, String generatorFolder);
    List<String> getGeneratorFolders(String problemFolder);
    CodeSupplier getGeneratorCodeSupplier(String problemFolder, String generatorFolder);
    
    // Методы для работы с папками валидаторов
    String addValidatorFolder(String problemFolder);
    void deleteValidatorFolder(String problemFolder, String validatorFolder);
    List<String> getValidatorFolders(String problemFolder);
    CodeSupplier getValidatorCodeSupplier(String problemFolder, String validatorFolder);

    // Методы для работы с папками авторских решений
    String addAuthorDecisionFolder(String problemFolder);
    void deleteAuthorDecisionFolder(String problemFolder, String authorDecisionFolder);
    List<String> getAuthorDecisionsFolderNames(String problemFolder);
    CodeSupplier getAuthorDecisionCodeSupplier(String problemFolder,
            String authorDecisionFolder);
    AuthorDecisionDescriptor getAuthorDecisionDescriptor(String problemFolder,
            String authorDecisionFolder);

    // Методы для работы с папками посылок
    String addSubmissionFolder();
    void deleteSubmissionFolder(String submissionFolder);
    List<String> getSubmissionsFolderNames();
    SubmissionDescriptor getSubmissionDescriptor(String submissionFolder);
    CodeSupplier getSubmissionCodeSupplier(String submissionFolder);

    // Методы для работы с временными файлами
    Path getTempFile();
    void deleteTempFile(Path path);
    void deleteAllTempFiles();

    // Метод доступа к папке с конфигурационными файлами
    Path getConfigurationFolder();
    
}
\end{verbatim}

\section*{\texttt{CodeFileSupplier.java}}
\begin{verbatim}
package com.ran.filesystem.supplier;

import java.nio.file.Path;

// Интерфейс поставщика программных файлов
public interface CodeSupplier {

    Path getFolder();
    Path getSourceFolder();
    Path getSourceFile();
    Path putSourceFile(Path sourceFile);
    Path getCompileFolder();
    Path getCompileFile();

}
\end{verbatim}

\section*{\texttt{EntityDescriptor.java}}
\begin{verbatim}
package com.ran.filesystem.descriptor;

// Импорт классов
// ...

// Класс для работы с XML-дескрипторами сущностей
public class EntityDescriptor {

    // Набор свойств сущности
    private Map<String, String> properties = new LinkedHashMap<>();
    // Путь к XML-файлу
    private Path path;
    // Имя корневого элемента в XML-файле
    private String rootNodeName;

    public EntityDescriptor(Path path, String rootNodeName) {
        this.path = path;
        this.rootNodeName = rootNodeName;
    }
    
    public String getProperty(String propertyName) {
        return properties.get(propertyName);
    }
    
    public void setProperty(String propertyName, String propertyValue) {
        properties.put(propertyName, propertyValue);
    }
    
    // Загрузка свойств из XML-файла
    public void load() {
        try {
            DocumentBuilderFactory factory = DocumentBuilderFactory.newInstance();
            DocumentBuilder builder = factory.newDocumentBuilder();
            Document document = builder.parse(path.toFile());
            NodeList nodeList = document.getDocumentElement().getChildNodes();
            for (int i = 0; i < nodeList.getLength(); i++) {
                Node node = nodeList.item(i);
                if (node instanceof Element) {
                    String propertyName = node.getNodeName();
                    String propertyValue = "";
                    if (node.getFirstChild() != null) {
                        propertyValue = node.getFirstChild().getNodeValue();
                    }
                    properties.put(propertyName, propertyValue);
                }
            }
        } catch (Exception exception) {
            FileSystemLogging.logger.log(Level.FINE,
                    "Exception while loading submission descriptor", exception);
        }
    }
    
    // Сохранение свойств в XML-файле
    public void persist() {
        try (OutputStream stream = Files.newOutputStream(path)) {
            DocumentBuilderFactory factory = DocumentBuilderFactory.newInstance();
            DocumentBuilder builder = factory.newDocumentBuilder();
            Document document = builder.newDocument();
            Node submissionElement = document.createElement(rootNodeName);
            for (Map.Entry<String, String> entry: properties.entrySet()) {
                Node newNode = document.createElement(entry.getKey());
                newNode.appendChild(document.createTextNode(entry.getValue()));
                submissionElement.appendChild(newNode);
            }
            document.appendChild(submissionElement);
            DOMImplementation implementation = document.getImplementation();
            DOMImplementationLS implementationLS = (DOMImplementationLS)implementation
                    .getFeature("LS", "3.0");
            LSSerializer serializer = implementationLS.createLSSerializer();
            serializer.getDomConfig().setParameter("format-pretty-print", true);
            LSOutput output = implementationLS.createLSOutput();
            output.setEncoding("UTF-8");
            output.setByteStream(stream);
            serializer.write(document, output);
        } catch (Exception exception) {
            FileSystemLogging.logger.log(Level.FINE,
                    "Exception while loading submission descriptor", exception);
        }
    }
    
    // Вспомогательные методы
    // ...
    
}
\end{verbatim}
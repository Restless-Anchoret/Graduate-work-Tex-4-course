В качестве языка программирования для реализации приложения выбран язык Java. Такой выбор объясняется удобством использования объектно-ориентирован\-ного программирования в данном языке и, следовательно, "--- удобством реализации паттернов проектирования~\cite{gamma}, которые будет уместно применить для расширяемости приложения.

При разработке приложения были использованы различные полезные технологии. Ниже приведён полный список использованных технологий:

\begin{itemize}
\item JDK 1.8~\cite{java} "--- выбрана версия 1.8, поскольку в ней поддерживаются лямбда-выражения, которые было весьма удобно использовать при реализации;
\item Swing "--- использован для реализации графического интерфейса;
\item Git "--- использован в качестве системы контроля версий;
\item GitHub "--- использован для хранения репозитория;
\item Maven "--- использован в качестве фреймворка для сборки проекта;
\item NetBeans 8.1 "--- данная IDE предоставляет как гибкие средства для редактирования и рефакторинга кода, так и удобный инструментарий для создания графического интерфейса при помощи библиотеки Swing.
\end{itemize}
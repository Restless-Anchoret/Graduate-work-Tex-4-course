\section*{\texttt{TestingSystem.java}}
\begin{verbatim}
package com.ran.testing.system;

// Интерфейс ядра тестирующей системы
public interface TestingSystem {
    
    public void start();
    public void addSubmission(TestingInfo info);
    public void stop();
    
}
\end{verbatim}

\section*{\texttt{MultithreadTestingSystem.java}}
\begin{verbatim}
package com.ran.testing.system;

// Импорт классов
// ...

// Реализация ядра тестирующей системы, работающая во многих потоках
public class MultithreadTestingSystem implements TestingSystem {

    private static final int DEFAULT_THREADS_QUANTITY = Runtime.getRuntime()
            .availableProcessors();
    private static MultithreadTestingSystem defaultSystem = null;
    
    public static MultithreadTestingSystem getDefault() {
        if (defaultSystem == null) {
            defaultSystem = new MultithreadTestingSystem(DEFAULT_THREADS_QUANTITY);
        }
        return defaultSystem;
    }
    
    // Максимальное количество используемых потоков
    private int threadsQuantity;
    // Поставщик временных и конфигурационных файлов
    private TestingFileSupplier fileSupplier;
    private ExecutorService service;
    private ExecutorCompletionService<TestingInfo> completionService;
    // Счётчик текущего количества проверяемых посылок
    private int submissionsCounter = 0;
    // Поток для обработки результатов проверки
    private Thread resultHandlersThread;

    // Конструкторы, геттеры, сеттеры
    // Синхронизованные методы обновления счётчика проверяемых посылок
    // ...
    
    // Инициализация и старт ядра тестирующей системы
    @Override
    public void start() {
        service = Executors.newFixedThreadPool(threadsQuantity);
        completionService = new ExecutorCompletionService<>(service);
        resultHandlersThread = new Thread(new ResultTracking());
        resultHandlersThread.start();
    }

    // Добавление новой посылки на проверку
    @Override
    public void addSubmission(TestingInfo info) {
        submissionsCounterUp();
        completionService.submit(new TestingTask(info, fileSupplier));
    }

    // Остановка ядра тестирующей системы
    @Override
    public void stop() {
        service.shutdown();
        resultHandlersThread.interrupt();
    }
    
    // Процесс обработки результатов проверки очередной посылки
    // Выполняется в отдельном потоке
    private class ResultTracking implements Runnable {
        public void run() {
            boolean interrupted = false;
            while (!interrupted || getSubmissionsCounter() > 0) {
                try {
                    Future<TestingInfo> future = completionService.take();
                    submissionsCounterDown();
                    TestingInfo info = future.get();
                    info.getTestResultHandler().process(info);
                } // Перехват исключений, вывод в лог
                // ...
            }
        }
    }
    
    // Задача тестирования одной посылки
    private static class TestingTask implements Callable<TestingInfo> {
        private TestingInfo info;
        private TestingFileSupplier fileSupplier;
        // Конструктор
        // ...
        public TestingInfo call() {
            info.getProblemTester().performTesting(fileSupplier, info);
            return info;
        }
    }
    
}
\end{verbatim}

\section*{\texttt{Verdict.java}}
\begin{verbatim}
package com.ran.testing.system;

// Виды вердиктов по посылке или тесту
public enum Verdict {

    NOT_TESTED,
    WAITING,
    TESTING,
    ACCEPTED,
    PRETESTS_ACC,
    PARTIAL_ACC,
    COMPILE_ERROR,
    RUNTIME_ERROR,
    WRONG_ANSWER,
    TIME_LIMIT,
    MEMORY_LIMIT,
    SECUR_VIOL,
    FAIL

}
\end{verbatim}

\section*{\texttt{VerdictInfo.java}}
\begin{verbatim}
package com.ran.testing.system;

// Полная информация о результате тестирования
public class VerdictInfo implements Cloneable {

    public static final VerdictInfo VERDICT_NOT_TESTED = new VerdictInfo(Verdict.NOT_TESTED);
    public static final VerdictInfo VERDICT_COMPILE_ERROR = new VerdictInfo(Verdict.COMPILE_ERROR);
    public static final VerdictInfo VERDICT_MEMORY_LIMIT = new VerdictInfo(Verdict.MEMORY_LIMIT);
    public static final VerdictInfo VERDICT_SECUR_VIOL = new VerdictInfo(Verdict.SECUR_VIOL);
    public static final VerdictInfo VERDICT_FAIL = new VerdictInfo(Verdict.FAIL);

    private Verdict verdict;
    private Integer decisionTime;
    private Short decisionMemory;
    private Short points;
    private Integer wrongTestNumber;

    // Конструкторы, геттеры, сеттеры
    // ...

    @Override
    public VerdictInfo clone() {
        return new VerdictInfo(this.verdict, this.points)
                .setDecisionTime(this.decisionTime)
                .setDecisionMemory(this.decisionMemory)
                .setWrongTestNumber(this.wrongTestNumber);
    }

}
\end{verbatim}

\section*{\texttt{TestingInfo.java}}
\begin{verbatim}
package com.ran.testing.system;

// Импорт классов
// ...

// Полная информация о том, как нужно тестировать очередную посылку
public class TestingInfo {

    // Обработчик результатов проверки посылки
    private TestResultHandler testResultHandler;
    // Алгоритм проверки посылки
    private ProblemTester problemTester;
    // Система оценивания
    private EvaluationSystem evaluationSystem;
    // Инструментарий языка программирования
    private LanguageToolkit languageToolkit;
    // Способ проверки корректности ответа по тесту
    private Checker checker;
    // Поставщик файлов с исходным кодом и скомпилированным файлом
    private CodeFileSupplier codeFileSupplier;
    // Поставщик файлов с тестами
    private ProblemFileSupplier problemFileSupplier;
    // Необходимость тестировать только на претестах
    private boolean pretestsOnly;
    // Ограничение по времени
    private Integer timeLimit;
    // Ограничение по памяти
    private Short memoryLimit;
    // Таблица с полной информацией о тестах из каждой группы
    private TestTable testTable;
    // Результаты тестирования посылки
    private VerdictInfo verdictInfo = null;

    // Конструктор, геттеры, сеттеры
    // ...

}
\end{verbatim}

\section*{\texttt{ProblemTester.java}}
\begin{verbatim}
package com.ran.testing.tester;

// Импорт классов
// ...

// Алгоритм тестирования посылки
public interface ProblemTester {

    void performTesting(TestingFileSupplier fileSupplier, TestingInfo info);

}
\end{verbatim}

\section*{\texttt{EvaluationSystem.java}}
\begin{verbatim}
package com.ran.testing.evaluation;

// Импорт классов
// ...

public interface EvaluationSystem {

    void orderTesting(TesterDelegate delegate, boolean pretestsOnly);
    VerdictInfo getVerdictInfo(TestTable testTable, boolean pretestsOnly);
    ProblemResult countProblemResult(TreeMap<Date, VerdictInfo> verdictsMap,
            Date competitionBeginning);

    // Интерфейс делегата для доступа к запуску тестирования
    public interface TesterDelegate {
        TestTable getTestTable();
        Integer performTestGroup(TestGroupType type, boolean upToFirstFailure);
        VerdictInfo performOneTest(TestGroupType type, int testNumber);
    }
    
    // Результат по определённой задаче
    public class ProblemResult {
        private short points;
        private int fine;
        // Конструктор, геттеры, сеттеры
        // ...
    }

}
\end{verbatim}

\section*{\texttt{LanguageToolkit.java}}
\begin{verbatim}
package com.ran.testing.language;

// Импорт классов
// ...

// Инструментарий языка программирования
public interface LanguageToolkit {
    
    int compile(Path sourceFile, Path compileFolder, Path configFolder) throws FailException;
    int compile(Path sourceFile, Path compileFolder, Path configFolder,
            OutputStream errorStream) throws FailException;
    ExecutionInfo execute(Path compileFile, Path inputFile, Path outputFile,
            Path configFolder, int timeLimit, short memoryLimit)
            throws FailException, TimeLimitException, MemoryLimitException,
            SecurityViolatedException;

    // Результат запуска программы
    public class ExecutionInfo {
        private int exitCode;
        private Integer decisionTime;
        private Short decisionMemory;
        // Конструктор, геттеры
        // ...
    }
    
    static final OutputStream EMPTY_OUTPUT_STREAM = new OutputStream() {
        @Override
        public void write(int b) throws IOException {
        }
    };
    
}
\end{verbatim}

\section*{\texttt{JavaLanguageToolkit.java}}
\begin{verbatim}
package com.ran.testing.language;

// Импорт классов
// ...

// Инструментарий для языка Java
public class JavaLanguageToolkit implements LanguageToolkit {
    
    private static final String POLICY_FILE_NAME = "java_problem.policy";
    private static final String PROPERTIES_FILE_NAME = "java.properties";
    private static final String JAVA_PATH_PROPERTY = "java_path";
    private static final String STACK_SIZE_PROPERTY = "stack_size";
    
    // Другие статические константы
    // ...
    
    private static Properties javaProperties = null;
    private static Path javaPropertiesLocation = null;
    
    // Метод доступа к заданным свойствам инструментария Java
    private static Properties getJavaProperties(Path configFolder) throws FailException {
        if (javaProperties == null || !Objects.equals(configFolder, javaPropertiesLocation)) {
            javaProperties = new Properties();
            try {
                javaProperties.load(Files.newInputStream(
                        configFolder.resolve(PROPERTIES_FILE_NAME)));
            } catch (IOException exception) {
                throw new FailException("Fail because of exception while loading "
                        + PROPERTIES_FILE_NAME, exception);
            }
            javaPropertiesLocation = configFolder;
        }
        return javaProperties;
    }
    
    // Компиляция программы на Java (без вывода ошибок)
    @Override
    public int compile(Path sourceFile, Path compileFolder, Path configFolder)
            throws FailException {
        return compile(sourceFile, compileFolder, configFolder, EMPTY_OUTPUT_STREAM);
    }

    // Компиляция программы на Java (с выводом ошибок)
    @Override
    public int compile(Path sourceFile, Path compileFolder, Path configFolder,
            OutputStream errorStream) throws FailException {
        if (Files.notExists(sourceFile) || Files.notExists(compileFolder)) {
            throw new FailException("Compilation failed because files were not found.");
        }
        JavaCompiler compiler = ToolProvider.getSystemJavaCompiler();
        return compiler.run(null, null, errorStream, "-d", compileFolder.toString(),
                sourceFile.toString());
    }

    // Запуск программы на Java
    @Override
    public ExecutionInfo execute(Path compileFile, Path inputFile, Path outputFile,
            Path configFolder, int timeLimit, short memoryLimit)
            throws FailException, TimeLimitException, MemoryLimitException,
            SecurityViolatedException {
        Properties javaProperties = getJavaProperties(configFolder);
        String javaPathProperty = javaProperties.getProperty(JAVA_PATH_PROPERTY);
        String stackSizeProperty = javaProperties.getProperty(STACK_SIZE_PROPERTY);
        Path policyPath = configFolder.resolve(POLICY_FILE_NAME);
        if (Files.notExists(compileFile) || Files.notExists(inputFile) ||
                Files.notExists(outputFile) || Files.notExists(policyPath) ||
                javaPathProperty == null || stackSizeProperty == null) {
            throw new FailException("Execution failed because files or properties were not found.");
        }
        try {
            Path javaPath = Paths.get(javaPathProperty);
            String memoryOption = "-Xmx" + memoryLimit + "M";
            String stackOption = "-Xss" + stackSizeProperty + "M";
            String managerOption = "-Djava.security.manager";
            String policyOption = "-Djava.security.policy==" + policyPath.toString();
            String className = getClassName(compileFile);
            ProcessBuilder processBuilder = new ProcessBuilder("\"" + javaPath.toString() + "\"",
                    memoryOption, stackOption, managerOption, policyOption, className);
            Path compileFolder = compileFile.getParent();
            processBuilder.directory(compileFolder.toFile());
            processBuilder.redirectInput(inputFile.toFile());
            processBuilder.redirectOutput(outputFile.toFile());
            Process process = null;
            try {
                long start = System.currentTimeMillis();
                process = processBuilder.start();
                process.waitFor(timeLimit, TimeUnit.MILLISECONDS);
                int decisionTime = (int)(System.currentTimeMillis() - start);
                if (process.isAlive()) {
                    throw new TimeLimitException(decisionTime);
                }
                int exitValue = process.exitValue();
                if (exitValue != 0) {
                    analyseErrorStream(process.getErrorStream());
                }
                return new ExecutionInfo(exitValue, decisionTime, null);
            } finally {
                if (process != null) {
                    process.destroyForcibly();
                    process.waitFor();
                }
            }
        } catch (InterruptedException | IOException exception) {
            throw new FailException(
                    "InterruptedException or IOException while execution.", exception);
        }
    }
    
    // Вспомогательные методы
    // ...

}
\end{verbatim}

\section*{\texttt{Checker.java}}
\begin{verbatim}
package com.ran.testing.checker;

// Импорт классов
// ...

// Способ проверки корректности ответа
public interface Checker {

    default void initialize(ProblemFileSupplier problemFileSupplier) { }
    Verdict check(Path inputPath, Path outputPath, Path answerPath);

}
\end{verbatim}
\input{preamble}

\begin{document}

\input{titul}

\chapter*{Аннотация}
Олимпиады по программированию сегодня "--- очень распространённое и актуальное явление. Они проводятся как среди школьников, так и среди студентов и профессиональных работников, как на уровне вузов, так и на международном уровне. Многие крупные IT-компании, такие как Google, Facebook, Яндекс, Mail.ru, ВКонтакте, регулярно проводят свои собственные олимпиады.

Проведение олимпиад по программированию требует наличия специального программного обеспечения: тестирующей системы, удовлетворяющей определённому набору требований, а также "--- системы для разработки задач. Подготовка наборов тестов для задач "--- весьма трудоёмкий процесс, требующий применения специальных инструментов, таких как генераторы, валидаторы, чекеры и интеракторы.

В данной работе ставится цель изучить основные аспекты работы тестирующих систем и систем разработки задач для олимпиад по программированию, а также "--- написать программу с удобным графическим пользовательским интерфейсом, предоставляющую возможность разрабатывать новые задачи и проверять решения на подготовленных тестах. Предполагается написание программы на языке Java с использованием библиотеки Swing.

\renewcommand{\contentsname}{Содержание}
\tableofcontents

\chapter*{Введение}
\addcontentsline{toc}{chapter}{Введение}

\input{introduction}

\chapter{Постановка задачи}
\section{Формулировка задачи}
Цель данной работы "--- написать программу с графическим пользовательским интерфейсом, предоставляющую определённые средства для разработки задач по олимпиадному программированию и для тестирования решений к созданным задачам. Рассмотрим основные варианты использования, которые необходимо реализовать. Так как разработку задач и тестирование решений можно рассматривать отдельно, варианты использования изображены ниже на двух отдельных диаграммах.

На рис.~\ref{use_case_diagram_development} изображена диаграмма вариантов использования, связанных с разработкой задач. Предполагается возможность прикреплять к задаче определённые параметры (такие, как ограничения по времени и памяти, условие задачи), а также "--- тесты, генераторы, валидаторы, чекеры и авторские решения (подробнее обо всём этом речь пойдёт в следующей главе).

\begin{figure}[h]
\center{\includegraphics[scale=0.8]{use_case_diagram_development}}
\caption{Варианты использования (разработка задач)}
\label{use_case_diagram_development}
\end{figure}

Создаваемое приложение должно давать возможность писать код для генераторов, валидаторов и чекеров прямо в своём графическом интерфейсе. Это предполагает также написание специальной библиотеки для разработки задач. Эта библиотека будет предоставлять методы с реализацией типичных операций, и пользователь сможет вызывать их прямо из своего кода. Кроме того, все перечисленные средства разработки можно будет компилировать и запускать из самого приложения.

Диаграмма на рис.~\ref{use_case_diagram_testing} отражает варианты использования, связанные с тестированием решений. Заметим здесь, что предполагается возможность отправлять решения, написанные на разных языках: в данной работе мы реализуем поддержу всего двух языков программирования "--- Java и C++. Также приложение должно будет поддерживать различные системы оценивания (их мы также подробно рассмотрим в следующей главе).

\begin{figure}[h]
\center{\includegraphics[scale=0.65]{use_case_diagram_testing}}
\caption{Варианты использования (тестирование)}
\label{use_case_diagram_testing}
\end{figure}

Приложение будет сохранять всю информацию в файловой системе и не будет использовать никаких баз данных. По определённому пути, который пользователь сможет выбрать, будет находиться определённая иерархия папок, в которых будут храниться все создаваемые и отправляемые файлы: условия задач, тесты, генераторы, валидаторы, чекеры, авторские и прочие решения, конфигурационные файлы. Все параметры, которые будут прикрепляться к задачам и решениям, будут также храниться в файловой системе в XML-дескрипторах.
\section{Средства реализации}
В качестве языка программирования для реализации приложения выбран язык Java. Такой выбор объясняется удобством использования объектно-ориентирован\-ного программирования в данном языке и, следовательно, "--- удобством реализации паттернов проектирования~\cite{gamma}, которые будет уместно применить для расширяемости приложения.

При разработке приложения были использованы различные полезные технологии. Ниже приведён полный список использованных технологий:

\begin{itemize}
\item JDK 1.8~\cite{java} "--- выбрана версия 1.8, поскольку в ней поддерживаются лямбда-выражения, которые было весьма удобно использовать при реализации;
\item Swing "--- использован для реализации графического интерфейса;
\item Git "--- использован в качестве системы контроля версий;
\item GitHub "--- использован для хранения репозитория;
\item Maven "--- использован в качестве фреймворка для сборки проекта;
\item NetBeans 8.1 "--- данная IDE предоставляет как гибкие средства для редактирования и рефакторинга кода, так и удобный инструментарий для создания графического интерфейса при помощи библиотеки Swing.
\end{itemize}

\chapter{Обзор существующих решений}
\section{Основные понятия}
\input{main_concepts}
\section{Разработка задач}
\input{development_info}
\section{Тестирующие системы}
\input{testing_info}

\chapter{Построение решения задачи}
\section{Функциональная декомпозиция}
Чтобы решить, каким образом мы будем организовывать структуру проекта, применим принцип функциональной декомпозиции. Для этого мы ещё раз вкратце рассмотрим, какие функции приложения нам нужно реализовать, и разделим их на несколько частей. Таким образом, мы определимся, на какие модули поделить проект.

Прежде всего, мы должны предоставить пользователю возможность писать свои генераторы, валидаторы и чекеры. Очевидно, для этого необходимо написать некоторую библиотеку с методами для решения типичных операций. Также для каждого средства разработки создадим свой абстрактный класс, который нужно будет расширить, реализовав всего один метод, чтобы получить готовый генератор, валидатор или чекер, способный выполнить свою задачу. Для этих средств разработки мы будем компилировать соответствующий класс и загружать его в виртуальная машину Java прямо во время выполнения.

Тестирование посылок логично выделяется в собственный модуль. Помимо реализации распараллеливания процесса тестирования этот модуль также включит в себя интерфейс для компиляции кода и выполнения программ, написанных на определённом языке, а также две реализации этого интерфейса "--- для языков Java и C++. Разумеется, использование интерфейса позволит в дальнейшем добавлять новые его реализации для поддержки новых языков. Следуя тому же принципу, в данный модуль добавим интерфейсы, соответствующие системе оценивания и способу проверки корректности ответа.

Для работы с файлами и организации иерархии папок в файловой системе также следует отвести отдельный модуль. В нём будут реализованы все типичные операции по сохранению, просмотру, изменению и удалению файлов. Для доступа к исходному коду и соответствующему скомпилированному файлу удобно ввести отдельный интерфейс, обеспечив к ним таким образом единообразный доступ. Также в данном модуле реализуем операции по чтению и записи XML-дескрипторов.

Остаётся только выделить отдельный модуль под графический интерфейс и бизнес-логику приложения. В него мы поместим все классы, соответствующие компонентам библиотеки Swing, и классы-контроллеры, реализующие все операции, выполняемые в ответ на действия пользователя. Также в этом модуле следует разместить вспомогательные классы, реализующие некоторые интерфейсы из других модулей.
\section{Архитектура}
Таким образом, мы определили структуру нашего проекта. Схематично она отражена на рис.~\ref{package_diagram_main}. Всего мы выделили четыре модуля:

\begin{itemize}
\item \texttt{Development} "--- библиотека для разработки задач;
\item \texttt{Testing} "--- ядро тестирующей системы;
\item \texttt{FileSystem} "--- работа с файлами;
\item \texttt{Interaction} "--- взаимодействие с пользоваталем.
\end{itemize}

\begin{figure}[h]
\center{\includegraphics[scale=0.8]{package_diagram_main}}
\caption{Структура проекта}
\label{package_diagram_main}
\end{figure}

Как можно видеть на диаграмме, модули связаны между собой зависимостями. Эти зависимости подразумевают собой обычное использование классов из другого модуля, то есть никакая инъекция зависимостей здесь не используется. Хотя идея о её использовании рассматривалась, было принято решение от неё отказаться, так как инъекция зависимостей не приносила особых преимуществ.

Каждый из модулей \texttt{Development}, \texttt{Testing} и \texttt{FileSystem} "--- полностью независим от всех остальных. Это означает, что любой из них можно безболезненно отделить от всего проекта и свободно использовать где-то ещё, как самостоятельную библиотеку простых Java-классов.

Впрочем, такой подход порождает ряд сложностей, поскольку, к примеру, модуль тестирования не может напрямую обращаться к методам доступа из модуля файловой системы, хотя должен производить непосредственную манипуляцию файлами. По той же причине в модуле со средствами разработки отсутствует возможность компилировать код, потому что код для компиляции находится в модуле тестирования.

Однако все эти проблемы легко решаются посредством введения интерфейсов-посредников. В модуле взаимодействия с пользователем, выступающим связующим звеном между остальными модулями, эти интерфейсы реализуются так, как того требует ситуация, и здесь уже не возникает проблем, потому что есть доступ к коду любого модуля.

\chapter{Описание практической части}
\section{Библиотека для разработки задач}
Библиотека для разработки задач.

\begin{figure}[h]
\center{\includegraphics[scale=0.8]{package_diagram_development}}
\caption{Модуль для разработки задач}
\label{package_diagram_development}
\end{figure}
\section{Тестирующий модуль}
Классы модуля тестирования также разделены на пакеты. Отдельный пакет отведён под классы, отвечающие за использование разных языков программирования, систем оценивания, способов проверки корректности ответа, за алгоритм тестирования, за распараллеливание процесса тестирования, а также за регистрирование объектов, доступных для использования в системе, и за логирование. Схема распределения классов в модуле тестирования отображена на рис.~\ref{package_diagram_testing}.

\begin{figure}[h]
\center{\includegraphics[scale=0.8]{package_diagram_testing}}
\caption{Модуль тестирования}
\label{package_diagram_testing}
\end{figure}

Центральное место в модуле занимают интерфейс \texttt{Testing\-System} и его реализация \texttt{Multithread\-Testing\-System}, которые принимают объекты класса \texttt{Tes\-ting\-Info} со всей необходимой информацией о посылке и ставят их в очередь на проверку. В классе \texttt{Multithread\-Testing\-System} происходит распараллеливание процесса проверки, благодаря чему каждая посылка проверяется в отдельном потоке. При этом на этапе инициализации возможно установить максимальное количество потоков, которым будет разрешено работать одновременно, так, чтобы появляющиеся новые посылки оставались в очереди и ожидали момента, когда для них освободится место. При инициализации объект этого класса также получает реализацию интерфейса \texttt{Testing\-File\-Supplier}, предоставляющего методы для работы с временными и конфигурационными файлами.

Также в системе запущен отдельный поток, в котором поочерёдно для каждой посылки после её проверки будут обрабатываться результаты посредством вызова заданной callback-функции (реализации интерфейса \texttt{Test\-Result\-Handler}). Поскольку этот процесс происходит в одном потоке, исключаются конфликты из-за конкурентного доступа в других модулях.

По окончании тестирования посылки ей присуждается некоторый вердикт с дополнительный информацией, которые записываются в объект класса \texttt{Verdict\-Info}. В нём хранится объект перечислимого типа \texttt{Verdict} (в нём перечислены все возможные вердикты, упомянутые в одной из предыдущих глав), а также максимальные время и память, затраченные решением за одном тесте, количество присуждённых очков и номер первого теста, который решение не прошло успешно, если это произошло. Такой же объект \texttt{Verdict\-Info} ставится в соответствие каждому тесту и записывается в объект класса \texttt{Test\-Table}, в котором хранится заранее подготовленная таблица с информацией о том, сколько тестов находится в каждой группе тестов.

Объект класса \texttt{Testing\-Info}, добавляемый в систему на проверку, должен быть подготовлен заранее один раз, и после отправки вызывающий код может больше не заботиться о нём. Важно только записать в него всю необходимую информацию для обработки посылки. Помимо некоторой общей информации о посылке (ограничения по времени и по памяти, которые нужно применить при тестировании, необходимости проверки только на претестах и подготовленного объекта класса \texttt{Test\-Table}) нужно также предоставить реализации следующих интерфейсов:

\begin{itemize}
\item \texttt{ProblemTester} "--- применяемый алгоритм проверки;
\item \texttt{EvaluationSystem} "--- применяемая система оценивания;
\item \texttt{LanguageToolkit} "--- выбранный язык программирования;
\item \texttt{Checker} "--- применяемый в задаче способ проверки правильности ответа;
\item \texttt{CodeFileSupplier} "--- предоставляет методы доступа к путям, по которым нужно искать файл с исходным кодом и располагать скомпилированный файл;
\item \texttt{ProblemFileSupplier} "--- предоставляет методы доступа к путям, по которым расположены тесты к задаче и скомпилированный файл чекера;
\item \texttt{TestResultHandler} "--- callback-функция для обработки результата тестирования посылки.
\end{itemize}

Рассмотрим первые четыре интерфейса более подробно. Начнём с интерфейса \texttt{Problem\-Tester}. В модуле предоставлена единственная его реализация "--- класс \texttt{Coding\-Problem\-Tester}, реализующий типичный алгоритм тестирования посылки на олимпиаде по программированию. Этот алгоритм, совмещённый с логикой системы оценивания ICPC, в общих чертах изображён на activity-диаграмме на рис.~\ref{activity_diagram_testing}. Он вполне согласуется с тем алгоритмов, что был описан в одной из предыдущих глав.

\begin{figure}[h]
\center{\includegraphics[scale=0.8]{activity_diagram_testing}}
\caption{Алгоритм тестирования посылки}
\label{activity_diagram_testing}
\end{figure}

Класс \texttt{Coding\-Problem\-Tester} общается с системой оценивания (реализацией интерфейса \texttt{Evaluation\-System}) посредством передачи ей объекта класса \texttt{Coding\-Tester\-Delegate} с методами для запуска тестирования посылки на определённом тесте, чтобы обеспечить системе оценивания возможность определить порядок проверки. Таким образом, поддерживается различный порядок запуска решения на тестах при разных системах оценивания.

Интерфейс \texttt{Evaluation\-System} имеет всего три метода: для определения порядка тестирования, для определения вердикта по посылке и для вычисления количества очков и штрафа по задаче на основе информации обо всех посылках, сделанных по данной задаче. В данном модуле присутствует три реализации этого интерфейса: \texttt{ICPC\-Evaluation\-System}, \texttt{IOI\-Evaluation\-System} (соответствующие двум реальным системам оценивания) и \texttt{Check\-Evaluation\-System} (специальная реализация для принудительной проверки посылки на всех тестах).

Интерфейс \texttt{Language\-Toolkit} содержит два метода: для компиляции исходного кода и для выполнения скомпилированного файла. Заметим, что именно на уровне реализаций данного интерфейса происходит отлавливание всех исключительных ситуаций, изображённых на рис.~\ref{activity_diagram_testing}. Присутствует две реализации "--- соответственно для языков Java и C++.

Класс \texttt{Java\-Language\-Toolkit} берёт из конфигурационного файла \texttt{java.pro\-perties} путь к установленному на компьютере JDK, а из файла \texttt{java\_problem.po\-licy} "--- список прав (пустой), присуждаемых запускаемым в отдельном процессе программам-решениям на языке Java. Компиляция Java-кода происходит с помощью специального API, имеющегося в JDK.

Класс \texttt{Visual\-Cpp\-Language\-Toolkit} для компиляции кода на C++ использует компилятор, поставляемый вместе со средой разработки Microsoft Visual Studio. Для этого используется конфигурационный файл \texttt{visual\_cpp.properties}. В нём прописывается путь к папке с компилятором и bat-файлом, который необходимо выполнить перед запуском компилятора. Как для компиляции, так и для выполнения скомпилированной программы запускается отдельный процесс.

Интерфейс \texttt{Checker} соответствует способу проверки правильности ответа участника на основе содержимого трёх файлов: входных данных, ответа участника и ответа жюри. По сути, он соответствует одноимённому классу из библиотеки для разработки задач, но напрямую с ним не связан (их связь реализована в модуле взаимодействия с пользователем). Зато в данном модуле есть реализация этого интерфейса по умолчанию "--- класс \texttt{MatchChecker}, "--- которая может использоваться в случаях, когда ответ в задаче единственен. Эта реализация просто проверяет на равенство ответ участника и эталонный ответ жюри.

Наконец, в данном модуле присутствует система реестров, реализующих интерфейс \texttt{Registry} и расширяющих абстрактный класс \texttt{AbstractRegistry}. Они нужны для того, чтобы хранить в них способы получения реализаций некоторых интерфейсов из данного модуля, поставленные в соответствие строковым идентификаторам. Таким образом, вызывающий код получает удобное решение в случае необходимости получить ту или иную реализацию.
\section{Модуль для работы с файловой системой}
Как говорилось ранее, приложение будет производить много работы с файлами, расположенными на диске. По этой причине удобно сформировать определённую иерархию папок, чтобы размещать в них файлы, а также "--- реализовать методы для удобного доступа к этим файлам.

Рассмотрим подробно иерархию папок, которая используется в нашем приложении. Корневая папка \texttt{file\_system} содержит четыре подпапки:

\begin{itemize}
\item \texttt{problems} "--- содержит пронумерованные папки со всеми файлами, относящимися к определённой задаче;
\item \texttt{submissions} "--- содержит пронумерованные папки с исходным кодом и скомпилированными файлами отправленных посылок;
\item \texttt{temp} "--- содержит временные файлы, периодически создаваемые и удаляемые приложением;
\item \texttt{config} "--- содержит конфигурационные файлы (необходимые, например, для корректной компиляции программ и их выполнения в модуле тестирования).
\end{itemize}

Каждая папка с посылкой содержит внутри себя XML-дескриптор, описывающий параметры посылки, и две папки "--- \texttt{src} c исходным кодом и \texttt{bin} со скомпилированным файлом. Сразу заметим, что способ хранения этих файлов аналогичен для авторских решений и средств разработки задач, что отражается на способе доступа к ним из программы.

Каждая папка с задачей так же содержит XML-дескриптор с параметрами задачи, условие (файл \texttt{statement.pdf}) и ещё несколько папок:

\begin{itemize}
\item \texttt{tests} "--- содержит по одной папке для каждой группы тестов, каждая из которых, в свою очередь, содержит XML-дескриптор группы тестов и пронумерованные папки с тестами (каждый тест представлен двумя файлами: \texttt{input.txt} и \texttt{output.txt});
\item \texttt{checker} "--- содержит две папки \texttt{src} и \texttt{bin} с файлами единственного чекера, прикреплённого к задаче;
\item \texttt{generators} "--- содержит набор пронумерованных папок с файлами генераторов (разложенных так же по папкам \texttt{src} и \texttt{bin});
\item \texttt{validators} "--- с пронумерованными папками для файлов валидаторов (разложенных по папкам \texttt{src} и \texttt{bin});
\item \texttt{author\_decisions} "--- с пронумерованными папками для файлов авторских решений (разложенных по папкам \texttt{src} и \texttt{bin}) и XML-дескриптором авторского решения.
\end{itemize}

Классы модуля для работы с файловой системы также разложены по пакетам. Отдельные пакеты отведены для классов, ответственных за иерархию папок, за обращение с XML-дескрипторами и за логирование. На рис.~\ref{package_diagram_file_system} отражена структура данного модуля.

\begin{figure}[h]
\center{\includegraphics[scale=0.8]{package_diagram_file_system}}
\caption{Модуль для работы с файловой системой}
\label{package_diagram_file_system}
\end{figure}

Интерфейс \texttt{File\-Supplier} предоставляет все методы, необходимые для работы с перечисленными файлами и папками, в том числе методы для их добавления, удаления и доступа к ним. Интерфейс устроен таким образом, что каждая задача, посылка, тест и любая другая сущность однозначно идентифицируется в своём контексте некоторым числовым идентификатором, соответствующим имени папки, в которой она расположена. Предоставляемая реализация интерфейса "--- класс \texttt{Standard\-File\-Supplier} "--- для выполнения однотипных операций с файлами использует множество статических методов вспомогательного класса \texttt{Files\-Util}. Также в момент инициализации класс \texttt{Standard\-File\-Supplier} принимает путь, по которому следует искать и располагать всю используемую иерархию папок.

Как уже говорилось выше, поскольку способ хранения исходного кода и скомпилированных программ одинаков для разных сущностей, для доступа к ним удобно добавить свой интерфейс. В роли такого интерфейса выступает \texttt{Code\-Supplier}, который можно получить, вызвав некоторые из методов \texttt{File\-Supplier}. Данный интерфейс не только даёт доступ к соответствующим файлам и папкам, но также даёт возможность положить в папку новый файл с исходный кодом. Используемая реализация этого интерфейса "--- класс \texttt{Standard\-Code\-Supplier}.

Некоторым из используемых сущностей также соответствуют некоторые свойства, которые хранятся в виде XML-файлов. Главный класс, реализующий взаимодействие с такими XML-дескрипторами "--- \texttt{Entity\-Descriptor}. Он хранит в себе объект типа \texttt{Map}, в котором ставятся в соответствие имена свойств и их значения. Помимо методов доступа и установки этих значений класс \texttt{Entity\-Descriptor} предоставляет также методы \texttt{load()} и \texttt{persist()} для автоматической загрузки и сохранения свойств из соответствующего файла. Для парсинга XML используется интерфейс DOM~\cite{cornell2}.

Каждой из сущностей, имеющих специальные свойства, соответствует некоторый класс, наследующий от \texttt{Entity\-Descriptor}. Всего таких классов четыре:

\begin{itemize}
\item \texttt{Problem\-Descriptor} "--- содержит название задачи, ограничения по времени и по памяти и используемый в задаче тип проверки корректности ответа;
\item \texttt{Submission\-Descriptor} "--- содержит идентификатор соответствующих задачи, системы оценивания и компилятора, а также вердикт по посылке, присуждённые очки, номер первого неудачного теста и использованные решением время и память;
\item \texttt{Author\-Decision\-Descriptor} "--- содержит название авторского решения и идентификатор компилятора;
\item \texttt{Test\-Group\-Descriptor} "--- содержит количество очков, присуждаемых за один тест из данной группы тестов.
\end{itemize}
\section{Модуль с графическим интерфейсом}
Схема распределения классов из модуля взаимодействия с пользователем по пакетам отображена на рис.~\ref{package_diagram_interaction}. Отдельные пакеты отведены под классы, соответствующие элементам графического интерфейса (компоненты Swing), классы-контроллеры с бизнес-логикой приложения, классы, реализующие паттерны проектирования <<Observer>> и <<Strategy>>, вспомогательные классы, а также под класс \texttt{Main} и класс для логирования. В этом разделе мы рассмотрим их подробно.

\begin{figure}[h]
\center{\includegraphics[scale=0.7]{package_diagram_interaction}}
\caption{Модуль взаимодействия с пользователем}
\label{package_diagram_interaction}
\end{figure}

В данном модуле для соединения графического интерфейса, бизнес-логики и отображаемых данных используется принцип схемы MVC. Это означает, что все классы, являющиеся элементами графического интерфейса, не содержат никакого кода бизнес-логики, а только оповещают классы-контроллеры о действиях со стороны пользователя. Классы-контроллеры, в свою очередь, выполняя некоторую бизнес-логику, берут из модели (в роли которой выступает интерфейс \texttt{File\-Supplier}) некоторую информацию и передают в компоненты Swing для отображения.

Таким образом, в данном модуле реализуется паттерн проектирования <<Ob\-ser\-ver>>~\cite{gamma}. Каждый класс, представляющий собой диалоговое окно, реализует интерфейс \texttt{Publisher}, имеющий методы для подписки в него по определённым идентификаторам реализаций интерфейса \texttt{Observer}. Как только в окне наступает некоторое событие, вызывается метод \texttt{notify()} соответствующего наблюдателя, и событие обрабатывается. При этом наблюдатели оформляют подписку на этапе инициализации окна. По такому принципу работают все диалоговые окна.

В стартовой точке приложения (в классе \texttt{Main}) создаётся экземпляр класса \texttt{Main\-Controller}, отображающий главное окно приложения \texttt{Main\-Frame}. В данном контроллере создаётся экземпляр класса \texttt{Problems\-Creator}, объединяющий в себе используемые повсюду в приложении объекты типов \texttt{Testing\-System} и \texttt{File\-Supplier}. Именно в данном классе инициализируются реализации этих интерфейсов. Из специального XML-дескриптора \texttt{creators-pa\-rams.xml} берутся два свойства: максимальное количество одновременно запущеных процессов в тестирующей системе и путь к папке в файловой системе для хранения иерархии папок.

Перечислим все окна, присутствующие в данном приложении.

\begin{itemize}
\item \texttt{MainFrame} "--- содержит глобальные списки сделанных посылок и созданных задач (на двух вкладках \texttt{Submissions\-Panel} и \texttt{Problems\-Panel}). Посылки можно перезапускать, а для задач "--- открывать диалог для редактирования. 
\item \texttt{ProblemDialog} "--- диалоговое окно редактирования одной задачи, с шестью вкладками: с общей информацией о задаче (\texttt{General\-Panel}), тестами (\texttt{Tests\-Panel}), генераторами (\texttt{Generators\-Panel}), валидаторами (\texttt{Validators\-Panel}), чекерами (\texttt{Checkers\-Panel}) и авторскими решениями (\texttt{Author\-Decisions\-Panel}).
\item \texttt{FileEditorDialog} "--- окно для просмотра содержимого файла. Может открываться в двух режимах: только для чтения или для редактирования. Используется для просмотра входных и выходных данных тестов и исходного кода.
\item \texttt{DevelopmentDialog} "--- окно для вывода результатов работы генератора или валидатора, а также "--- процесса создания ответов на тесты.
\item \texttt{AddingAuthorDecisionDialog} "--- окно для добавления нового решения.
\item \texttt{SubmissionCreationDialog} "--- окно для выбора параметров новой посылки.
\item \texttt{SubmissionResultDialog} "--- окно с выводом полных результатов посылки или авторского решения на каждом тесте.
\end{itemize}

Заметим, что для реализации работы контроллера \texttt{Development\-Controller} был применён паттерн проектирования <<Strategy>>~\cite{gamma}, позволивший задавать контроллеру ожидаемое поведение. Для этого был отведён пакет с набором стратегий: для запуска генератора, валидатора и процесса создания ответов на тесты.

Также в отдельный пакет были вынесены дополнительные компоненты Swing, которых не хватало в библиотеке. Это, к примеру, текстовое поле, совмещённое с кнопкой <<Browse>> для удобного ввода пути к файлу, и выпадающий список с сопоставлением каждому значению удобочитаемого отображения в виде строки.

И, наконец, есть пакет со всеми вспомогательными классами, необходимыми в том числе для связывания воедино всех остальных модулей. Здесь находятся реализации интерфейсов из модуля тестирования, делегирующие выполнение интерфейсу \texttt{File\-Supplier}; классы \texttt{Swing\-Util} и \texttt{Testing\-Util}, реализующие некоторые типичные операции; класс \texttt{Java\-Class\-Checker}, позволяющий проверять ответ участника с помощью класса чекера, загруженного во время выполнения; класс \texttt{Test\-Table\-Wrapper} для облегчения работы с группами тестов; и класс \texttt{Presentation\-Support} с методами для определения визуального представления некоторых строковых идентификаторов.
\section{Пример работы приложения}
Рассмотрим работу нашего приложения на примере подготовки такой простой задачи: необходимо для некоторого целого числа $n$ ($2 \leq n \leq 10^{12}$) определить, является ли оно простым, и если является, вывести -1, а в противном случае "--- произвольный его делитель, отличный от единицы и самого числа $n$. Заметим, что в задаче правильный ответ может быть не единственным, поэтому необходимо писать код для чекера. Также мы рассмотрим процесс генерации и валидации тестов, прикрепления авторского решения к задаче и отправки посылки.

На рис.~\ref{screen_problems} отображена вкладка главного окна программы со списком всех созданных задач. Здесь можно добавить или удалить задачу, а также перейти к её редактированию. Как можно видеть, задача <<Prime checking>> уже добавлена.

\begin{figure}[h]
\center{\includegraphics[scale=0.9]{screen_problems}}
\caption{Список созданных задач}
\label{screen_problems}
\end{figure}

Если выбрать задачу и нажать на кнопку <<Edit>>, откроется новое окно с шестью вкладками для редактирования выбранной задачи. На рис.~\ref{screen_problem_param} показана вкладка с основными параметрами задачи: названием, ограничениями по времени и памяти. Также здесь можно загрузить новый файл с условием. Мы выберем ограничения, равные двум секундам и 64-м мегабайтам.

На следующей вкладке (рис.~\ref{screen_tests}) можно посмотреть списки тестов, распределённые по группам, которые можно выбирать в выпадающем списке. Здесь можно добавить или удалить тест, просмотреть входные и выходные данные теста, чтобы изменить их вручную, а также "--- указать количество очков, присуждаемое за один тест из данной группы.

\begin{figure}[h]
\center{\includegraphics[scale=0.9]{screen_problem_param}}
\caption{Основные параметры задачи}
\label{screen_problem_param}
\end{figure}

\begin{figure}[h]
\center{\includegraphics[scale=0.9]{screen_tests}}
\caption{Группа тестов в задаче}
\label{screen_tests}
\end{figure}

\begin{figure}[!ht]
\center{\includegraphics[scale=1.0]{screen_generators}}
\caption{Указание параметров запуска генератора}
\label{screen_generators}
\end{figure}

На вкладке <<Generators>> (рис.~\ref{screen_generators}) можно просмотреть список созданных для этой задачи генераторов, добавить новый (при этом копируется пустой шаблон для генератора) или удалить старый, а также изменить код. На этой же странице можно запускать генератор. Для этого нужно выбрать группу тестов, в которую затем нужно будет сохранить сгенерированные файлы, количество создаваемых тестов, инициализирующее значение для генератора псевдослучайных чисел и строковые параметры генератора, разделённые пробелом.

При нажатии на кнопку <<View code>> открывается окно, отображённое на рис.~\ref{screen_generator_code}. Такое же окно открывается для просмотра и изменения содержимого всех остальных файлов. В данном случае в нём доступен режим редактирования. Для данной задачи достаточно простейшего генератора, принимающего два числовых параметра для определения границ диапазона, в котором должно находиться число $n$ из входных данных, и генерирующего это число.

На рис.~\ref{screen_generator_result} показано окно, которое появится, если мы запустим генератор для создания 15-ти новых тестов для группы <<Tests 2>> с диапазоном для числа $n$ от миллиарда до триллиона. В данном окне отображаются сообщения о ходе процесса генерации тестов, в том числе "--- сколько времени в миллисекундах было потрачено на генерацию каждого теста. Для сохранения сгенерированных тестов в задаче нужно дополнительно нажать на кнопку <<Save>>. Кроме того, можно преждевременно нажать на кнопку <<Interrupt>>, чтобы прервать процесс генерации, если он выполняется слишком долго.

\begin{figure}[!p]
\center{\includegraphics[scale=1.0]{screen_generator_code}}
\caption{Код генератора}
\label{screen_generator_code}
\end{figure}

\begin{figure}[!p]
\center{\includegraphics[scale=1.0]{screen_generator_result}}
\caption{Результаты запуска генератора}
\label{screen_generator_result}
\end{figure}

Чтобы просмотреть список прикреплённых к задаче валидаторов, нужно перейти на вкладку <<Validators>>. Здесь так же можно добавить и удалить валидатор, открыть его для редактирования или запустить, указав группу тестов для валидации (или специальный пункт <<All tests>>) и строковые параметры запуска. Данная вкладка отображена на рис.~\ref{screen_validators}. Для данной задачи валидатор также прост: происходит проверка принадлежности числа $n$ диапазону от 2 до $10^{12}$.

На вкладке <<Checkers>> (рис.~\ref{screen_checkers}) можно выбрать тип чекера, используемого при проверке решения данной задачи: либо чекер по умолчанию (<<Matching with the answer>>), либо специальный чекер с написанным для него кодом (<<Special checker>>). Поскольку к одной задаче можно прикрепить только один чекер, здесь не отображается список чекеров. Также отсюда можно открыть окно для редактирования кода чекера и попробовать перекомпилировать его, чтобы удостовериться, что код написан корректно.

\begin{figure}[h]
\center{\includegraphics[scale=0.9]{screen_validators}}
\caption{Указание параметров запуска валидатора}
\label{screen_validators}
\end{figure}

\begin{figure}[h]
\center{\includegraphics[scale=0.9]{screen_checkers}}
\caption{Указание типа чекера}
\label{screen_checkers}
\end{figure}

Чекер, используемый в данной задаче, выполняет следующие действия. Если участник вывел -1, чекер проверяет, что в ответе жюри тоже находится -1. Если же участник вывел целое число в интервале от 2 до $n$, проверяется, является ли это число множителем $n$.

\begin{figure}[!ht]
\center{\includegraphics[scale=0.9]{screen_author_decisions}}
\caption{Добавление нового авторского решения}
\label{screen_author_decisions}
\end{figure}

На вкладке <<Author decisions>> можно просмотреть список авторских решений, прикреплённых к задаче (рис.~\ref{screen_author_decisions}). Здесь можно добавить новое решение (при этом отображается маленькое окно для ввода названия решения и выбора компилятора и файла с исходным кодом), удалить решение, открыть его код или запустить, чтобы увидеть вердикт на каждом тесте задачи. Кроме того, кнопка <<Create answers>> запускает генерацию ответов жюри (на каждом тесте из определённой группы, либо на тестах из всех групп, если выбран пункт <<All tests>>).

Главная идея авторского решения состоит в том, чтобы перечислять не все числа от 2 до $n$ для проверки, являются ли они множителем $n$, а только числа от 2 до $\sqrt{n}$, что, очевидно, происходит гораздо быстрее. Решение данной задачи на языке C++ состоит всего из одной функции \texttt{main()}, код которой выглядит следующим образом.

{\small
\begin{verbatim}
long long n;
cin >> n;

int s = (int)ceil(sqrt((double)n));
for (int i = 2; i <= s; i++) {
    if (i != n && n % i == 0) {
        cout << i << endl;
        return 0;
    }
}

cout << -1 << endl;
return 0;
\end{verbatim}
}

\begin{figure}[!b]
\center{\includegraphics[scale=0.9]{screen_submissions}}
\caption{Список отправленных посылок}
\label{screen_submissions}
\end{figure}

\begin{figure}[!b]
\center{\includegraphics[scale=0.9]{screen_new_submission}}
\caption{Выбор параметров новой посылки}
\label{screen_new_submission}
\end{figure}

Рассмотрим теперь вторую вкладку главного окна приложения (рис.~\ref{screen_submissions}), на которой располагается список всех сохранённых в файловой системе посылок с вердиктами по каждой из них. Здесь можно добавить новую посылку, при этом открывается новое окно, отображённое на рис.~\ref{screen_new_submission}, в котором можно выбрать задачу, компилятор, систему оценивания и файл с исходный кодом. Также на данной вкладке можно удалить посылку, открыть её код для просмотра и перезапустить.

Мы сделаем посылку с решением, написанным на языке Java, но отличающимся от предыдущего рассмотренного решения в способе вывода множителя числа $n$. Если в прошлый раз мы выводили первое найденное число $i$, на которое $n$ делится нацело, то теперь мы будем выводить $n/i$. Ответ всё равно останется правильным, но во многих случаях окажется другим.

Запустим посылку и увидим окно, отображённое на рис.~\ref{screen_submission_results}, где видны вердикт и затраченное время по каждому тесту в задаче. Как можно видеть, такое решение также даёт вердикт <<Accepted>> на всех тестах, что говорит о корректной работе написанного нами чекера.

\begin{figure}[h]
\center{\includegraphics[scale=0.9]{screen_submission_results}}
\caption{Результаты запуска посылки на каждом тесте}
\label{screen_submission_results}
\end{figure}

\chapter*{Заключение}
\addcontentsline{toc}{chapter}{Заключение}
В данной работе была поставлена задача написать приложение для разработки задач по олимпиадному программированию и для тестирования решений. Предполагалось использование языка программирования Java и библиотеки компонентов графического интерфейса Swing.

Для решения поставленной задачи были рассмотрены основные принципы разработки задач и тестирования решений. Были описаны правила написания кода таких средств, как генераторы, валидаторы и чекеры, рассмотрены алгоритм тестирования посылки, различные системы оценивания и используемые вердикты, присуждаемые решениям.

Была спроектирована архитектура приложения, которое включило четыре модуля: библиотеку для разработки задач, а также модули для тестирования, доступа к файловой системе и взаимодействия с пользователем. Приложение было разработано, и практически все пакеты и классы, вошедшие в него, были подробно описаны. Наконец, был приведён пример работы приложения, описывающий процесс разработки простой задачи.

Ясно, что тематика, затронутая в данной работе, весьма актуальна сегодня, поэтому разработанное приложение вполне может найти себе применение, оказав большую помощь при подготовке тестов к задачам. Кроме того, функционал данного приложения возможно расширять и улучшать.

%В ходе курсовой работы было исследовано четыре различных алгоритма поиска выравниваний пар строк, обладающих определёнными характеристиками. Мы подробно обсудили, как используется в них принцип динамического программирования и каким образом строится искомое оптимальное выравнивание, поговорили об эффективности каждого из этих алгоритмов, а также дополнительно для двух из них выявили способ нахождения количества кооптимальных выравниваний.

%Была поставлена задача написать приложение, реализующее все четыре алгоритма и позволяющее запускать их с некоторым набором входных параметров. В ходе работы это приложение было разработано, в процессе чего активно использовались объектно-ориентированное программирование и паттерны проектирования. Была подробно описана структура классов, использованных в приложении, и приведён пример его работы.

%Ясно, что в созданном приложении реализованы лишь некоторые алгоритмы, работающие с выравниваниями строковых последовательностей, и что помимо них существует ещё множество подобных алгоритмов, из чего следует возможность дальнейшей разработки данного приложения.

\renewcommand{\bibname}{Список использованных источников}
\begin{thebibliography}{99}
\addcontentsline{toc}{chapter}{Список использованных источников}
\bibitem{wiki} Олимпиады по программированию "--- Википедия.~"--- (URL: https://ru. wikipedia.org/wiki/Олимпиады\_по\_программированию).
\bibitem{ioi} Международная олимпиада по информатике "--- Википедия.~"--- (URL: https:// ru.wikipedia.org/wiki/Международная\_олимпиада\_по\_информатике).
\bibitem{icpc} Международная студенческая олимпиада по программированию "--- Википедия.~"--- (URL: https://ru.wikipedia.org/wiki/Международная\_студенческая\_ олимпиада\_по\_про\-грам\-ми\-рованию).
\bibitem{polygon} Платформа Polygon: [сайт].~"--- (URL: https://polygon.codeforces.com/).
\bibitem{codeforces} Платформа Codeforces: [сайт].~"--- (URL: http://codeforces.com/).
\bibitem{testlib} Testlib "--- Codeforces.~"--- (URL: http://codeforces.com/testlib).
\bibitem{gamma} Приёмы объектно-ориентированного проектирования. Паттерны проектирования~/ Гамма Э. [и др.].~"--- СПб.~: Питер, 2001.~"--- 368 с.
\bibitem{java} Java Platform Standard Edition 8 Documentation: [сайт].~"--- (URL: http:// docs.oracle.com/javase/8/docs/).
\bibitem{cornell1} Хорстманн, Кей С. Java. Библиотека профессионала, том 1. Основы~/ Хорстманн, Кей С., Корнелл, Гари.~"--- 9-е изд.: Пер. с англ.~"--- М.~: ООО~<<И.Д.~Вильямс>>, 2014.~"--- 864 с.
\bibitem{cornell2} Хорстманн, Кей С. Java. Библиотека профессионала, том 2. Расширенные средства~/ Хорстманн, Кей С., Корнелл, Гари.~"--- 9-е изд.: Пер. с англ.~"--- М.~: ООО~<<И.Д.~Вильямс>>, 2014.~"--- 1008 с.
\end{thebibliography}

\newpage
\appendix
\addtocontents{toc}{\vspace{5mm}\bfseries Приложения\par}
\footnotesize
\chapter{Библиотека для разработки задач}
\section*{\texttt{Generator.java}}
\begin{verbatim}
package com.ran.development.gen;

// Импорт классов

public abstract class Generator {

    private static final long DEFAULT_RANDOM_SEED = 1938572538L;
    
    public static Generator getDefault() {
        return new Generator() {
            @Override
            protected void generate(String[] args) { }
        };
    }
    
    private Random random;
    private PrintStream output;
    private RegularExpressionParser parser = new RegularExpressionParser();
    
    // Контструкторы, геттеры, сеттеры
    
    // print-методы
    
    protected int nextInt(int higherBound) {
        return random.nextInt(higherBound);
    }
    
    protected int nextInt(int lowerBound, int higherBound) {
        return lowerBound + random.nextInt(higherBound - lowerBound + 1);
    }
    
    // Ещё next-методы
    
    protected int any(int[] values) {
        return values[random.nextInt(values.length)];
    }
    
    protected long any(long[] values) {
        return values[random.nextInt(values.length)];
    }
    
    // Ещё any-методы
    
    protected char any(String regularExpression) {
        int quantity = parser.getCharactersQuantity(regularExpression);
        return parser.getCharacter(regularExpression, random.nextInt(quantity));
    }
    
    protected String anyLine(String regularExpression, int length) {
        return anyLines(regularExpression, length, 1)[0];
    }
    
    protected String[] anyLines(String regularExpression, int lengthOfLine, int linesQuantity) {
        int quantity = parser.getCharactersQuantity(regularExpression);
        String[] stringsArray = new String[linesQuantity];
        char[] array = new char[lengthOfLine];
        for (int i = 0; i < linesQuantity; i++) {
            for (int j = 0; j < lengthOfLine; j++) {
                array[j] = parser.getCharacter(regularExpression, random.nextInt(quantity));
            }
            stringsArray[i] = new String(array);
        }
        return stringsArray;
    }
    
    abstract protected void generate(String[] args);
    
}
\end{verbatim}

\section*{\texttt{MultiGenerator.java}}
\begin{verbatim}
package com.ran.development.gen;

// Импорт классов

public class MultiGenerator {
    
    private static final long MESSAGING_DELAY = 2000;
    
    private DevelopmentListenerSupport listenerSupport = new DevelopmentListenerSupport();
    private Supplier<? extends Generator> generatorSupplier = () -> {
        return Generator.getDefault();
    };
    private int randomSeed = 0;
    private Path[] paths = { };
    private String[] arguments = { };
    
    // Геттеры, сеттеры, методы работы со слушателями
    
    public void performGenerating() {
        listenerSupport.fireProcessingStarted();
        Generator generator = generatorSupplier.get();
        if (generator == null) {
            finishEarlier("Cannot instantiate Generator subclass",
                    DevelopmentResult.FAIL, 1, paths.length);
            return;
        }
        generator.setRandomSeed(randomSeed);
        for (int index = 1; index <= paths.length; index++) {
            listenerSupport.fireTaskProcessingStarted(index);
            Path path = paths[index - 1];
            long start = System.currentTimeMillis();
            Thread thread = null;
            try (OutputStream output = Files.newOutputStream(path)) {
                generator.setOutput(output);
                FutureTask<DevelopmentResult> futureTask = new FutureTask<>(
                        new GeneratingTask(generator, index, arguments));
                thread = new Thread(futureTask);
                start = System.currentTimeMillis();
                thread.start();
                while (thread.isAlive()) {
                    thread.join(MESSAGING_DELAY);
                    if (thread.isAlive()) {
                        listenerSupport.fireTaskIsProcessing(index,
                                System.currentTimeMillis() - start);
                    }
                }
                listenerSupport.fireTaskIsDone(futureTask.get());
            } // Перехват исключений, вывод в лог
        }
        listenerSupport.fireProcessingFinished();
    }
    
    // Вспомогательные методы
    
    private static class GeneratingTask implements Callable<DevelopmentResult> {
        private String[] arguments;
        private Generator generator;
        private int generatorNumber;
        
        // Конструктор
        
        @Override
        public DevelopmentResult call() {
            long start = System.currentTimeMillis();
            try {
                generator.generate(arguments);
            } // Перехват исключений, вывод в лог
            return new DevelopmentResult(generatorNumber, System.currentTimeMillis() - start);
        }
    }
    
}
\end{verbatim}

\section*{\texttt{Validator.java}}
\begin{verbatim}
package com.ran.development.valid;

// Импорт классов

public abstract class Validator {

    public static Validator getDefault() {
        return new Validator() {
            @Override
            public void validate(String[] args) throws ValidationException { }
        };
    }
    
    private StrictInput input = StrictInput.getEmptyInput();
    private RegularExpressionParser parser = new RegularExpressionParser();

    // Конструкторы, геттеры, сеттеры
    
    protected void error() throws ValidationException {
        throw new ValidationException(input.getState());
    }
    
    protected void error(String message) throws ValidationException {
        throw new ValidationException(message, input.getState());
    }
    
    protected void ensure(boolean condition) throws ValidationException {
        if (!condition) {
            error("Condition is not true");
        }
    }
    
    protected void inBounds(int lowEdge, int number, int highEdge)
            throws ValidationException {
        if (!(lowEdge <= number && number <= highEdge)) {
            error("Number out of bounds");
        }
    }
    
    // Ещё inBounds-методы
    
    protected void matchesExpression(char symbol, String regularExpression)
            throws ValidationException {
        if (!parser.matchesExpression(symbol, regularExpression)) {
            error("Symbol does not match the regular expression");
        }
    }
    
    protected void matchesExpression(String line, String regularExpression)
            throws ValidationException {
        if (!parser.matchesExpression(line, regularExpression)) {
            error("Line does not match the regular expression");
        }
    }
    
    protected void matchesExpression(String[] lines, String regularExpression)
            throws ValidationException {
        for (String line: lines) {
            if (!parser.matchesExpression(line, regularExpression)) {
                error("Line does not match the regular expression");
            }
        }
    }
    
    protected void inValues(int number, int[] values) throws ValidationException {
        for (int value: values) {
            if (value == number) {
                return;
            }
        }
        error("Number is not in values array");
    }
    
    // Ещё inValues-методы
    
    public void performValidation(String[] args) throws ValidationException {
        validate(args);
        if (!input.isReaden()) {
            error("File was not readen completely");
        }
    }
    
    public abstract void validate(String[] args) throws ValidationException;
    
}
\end{verbatim}

\section*{\texttt{MultiValidator.java}}
\begin{verbatim}
package com.ran.development.valid;

// Импорт классов

public class MultiValidator {

    private static final long MESSAGING_DELAY = 2000;
    
    private DevelopmentListenerSupport listenerSupport = new DevelopmentListenerSupport();
    private Supplier<? extends Validator> validatorSupplier = () -> {
        return Validator.getDefault();
    };
    private Path[] paths = { };
    private String[] arguments = { };
    
    // Геттеры, сеттеры, методы работы со слушателями
    
    public void performValidating() {
        listenerSupport.fireProcessingStarted();
        Validator validator = validatorSupplier.get();
        if (validator == null) {
            finishEarlier("Cannot instantiate Validator subclass",
                    DevelopmentResult.FAIL, 1, paths.length);
            return;
        }
        for (int index = 1; index <= paths.length; index++) {
            listenerSupport.fireTaskProcessingStarted(index);
            Path path = paths[index - 1];
            long start = System.currentTimeMillis();
            Thread thread = null;
            try (InputStream inputStream = Files.newInputStream(path)) {
                validator.setInputStream(inputStream);
                FutureTask<DevelopmentResult> futureTask = new FutureTask<>(
                        new ValidatingTask(validator, index, arguments));
                thread = new Thread(futureTask);
                start = System.currentTimeMillis();
                thread.start();
                while (thread.isAlive()) {
                    thread.join(MESSAGING_DELAY);
                    if (thread.isAlive()) {
                        listenerSupport.fireTaskIsProcessing(index,
                                System.currentTimeMillis() - start);
                    }
                }
                listenerSupport.fireTaskIsDone(futureTask.get());
            } // Перехват исключений, вывод в лог
        }
        listenerSupport.fireProcessingFinished();
    }
    
    // Вспомогательные методы
    
    private static class ValidatingTask implements Callable<DevelopmentResult> {
        private String[] arguments;
        private Validator validator;
        private int validatorNumber;
        
        // Конструктор
        
        @Override
        public DevelopmentResult call() {
            long start = System.currentTimeMillis();
            try {
                validator.performValidation(arguments);
            } catch (ValidationException exception) {
                StrictInput.State state = exception.getStrictInputState();
                String message = exception.getMessage();
                if (!state.isReaden()) {
                    message += " (line = " + state.getLineNumber() + ", column = " +
                            state.getPosition() + ")";
                }
                return new DevelopmentResult(validatorNumber, message, DevelopmentResult.FAIL,
                        System.currentTimeMillis() - start);
            } // Перехват исключений, вывод в лог
            return new DevelopmentResult(validatorNumber, System.currentTimeMillis() - start);
        }
    }
    
}
\end{verbatim}

\section*{\texttt{Checker.java}}
\begin{verbatim}
package com.ran.development.check;

// Импорт классов

public abstract class Checker {

    public static final int OK = 0,
            WRONG_ANSWER = 1,
            FAIL = 2;

    private TokenInput input = TokenInput.getEmptyInput();
    private TokenInput output = TokenInput.getEmptyInput();
    private TokenInput answer = TokenInput.getEmptyInput();
    private RegularExpressionParser parser = new RegularExpressionParser();

    // Геттеры, сеттеры

    protected void quit(int resultInfo, String message) throws CheckResultException {
        throw new CheckResultException(resultInfo, message);
    }

    // Ещё quit-методы

    protected void quitIf(boolean condition, int resultInfo, String message)
            throws CheckResultException {
        if (condition) {
            quit(resultInfo, message);
        }
    }

    // Ещё quitIf-методы

    protected void ensureIf(boolean condition, String message) throws CheckResultException {
        if (!condition) {
            quit(WRONG_ANSWER, message);
        }
    }

    // Ещё ensureIf-методы

    public void performChecking() throws CheckResultException {
        try {
            check();
        } catch (CheckResultException exception) {
            if (exception.getResultInfo() == OK && !output.checkEof()) {
                quit(WRONG_ANSWER, "Extra values in the output");
            }
            throw exception;
        }
        quit(FAIL, "Checker did not tell result of checking");
    }

    public abstract void check() throws CheckResultException;

}
\end{verbatim}

\section*{\texttt{WrapperChecker.java}}
\begin{verbatim}
package com.ran.development.check;

// Импорт классов

public class WrapperChecker {

    private static final long DEFAULT_DELAY_TIME = 10_000;
    private static final long MESSAGING_DELAY = 2000;

    private Supplier<? extends Checker> checkerSupplier;
    private Checker checker = null;
    private DevelopmentListenerSupport listenerSupport = new DevelopmentListenerSupport();
    private long delayTime;

    // Конструкторы, методы работы со слушателями

    public int check(Path inputPath, Path outputPath, Path answerPath) {
        listenerSupport.fireTaskProcessingStarted(0);
        if (checker == null) {
            checker = checkerSupplier.get();
        }
        if (checker == null) {
            String message = "Cannot load Checker subclass";
            DevelopmentLogging.logger.fine(message);
            listenerSupport.fireTaskIsDone(new DevelopmentResult(0, message, Checker.FAIL));
            return Checker.FAIL;
        }
        long start = System.currentTimeMillis();
        Thread thread = null;
        try (InputStream input = Files.newInputStream(inputPath);
                InputStream output = Files.newInputStream(outputPath);
                InputStream answer = Files.newInputStream(answerPath)) {
            checker.setInput(input);
            checker.setOutput(output);
            checker.setAnswer(answer);
            FutureTask<CheckingResult> futureTask = new FutureTask<>(new CheckingTask(checker));
            thread = new Thread(futureTask);
            start = System.currentTimeMillis();
            thread.start();
            while (thread.isAlive()) {
                thread.join(MESSAGING_DELAY);
                if (thread.isAlive()) {
                    long time = System.currentTimeMillis() - start;
                    if (delayTime > 0 && time > delayTime) {
                        thread.stop();
                        String message = "Checker is working longer than expected";
                        DevelopmentLogging.logger.fine(message);
                        listenerSupport.fireTaskIsDone(new DevelopmentResult(0, message,
                                DevelopmentResult.FAIL, time));
                        return Checker.FAIL;
                    }
                    listenerSupport.fireTaskIsProcessing(0, time);
                }
            }
            CheckingResult checkingResult = futureTask.get();
            if (checkingResult.getCheckingInfo() == Checker.FAIL) {
                DevelopmentLogging.logger.fine(checkingResult.getDevelopmentResult()
                        .getMessage());
            }
            listenerSupport.fireTaskIsDone(checkingResult.getDevelopmentResult());
            return checkingResult.getCheckingInfo();
        } // Перехват исключений, вывод в лог
        return Checker.FAIL;
    }

    private static class CheckingResult {
        private DevelopmentResult developmentResult;
        private int checkingInfo;
        // Конструкторы, геттеры
    }

    private static class CheckingTask implements Callable<CheckingResult> {
        private Checker checker;
        // Конструктор

        @Override
        public CheckingResult call() {
            long start = System.currentTimeMillis();
            try {
                checker.performChecking();
            } catch (CheckResultException exception) {
                int resultInfo = exception.getResultInfo();
                int developmentResultInfo = (resultInfo == Checker.FAIL
                        ? DevelopmentResult.FAIL : DevelopmentResult.OK);
                DevelopmentResult developmentResult = new DevelopmentResult(
                        0, exception.getMessage(), developmentResultInfo,
                        System.currentTimeMillis() - start);
                return new CheckingResult(developmentResult, resultInfo);
            } // Перехват исключений, вывод в лог
            String message = "Checker did not throw any expected exception while its work";
            DevelopmentLogging.logger.fine(message);
            DevelopmentResult developmentResult = new DevelopmentResult(0, message,
                    DevelopmentResult.FAIL, System.currentTimeMillis() - start);
            return new CheckingResult(developmentResult, Checker.FAIL);
        }
    }

}
\end{verbatim}

\section*{\texttt{Utils.java}}
\begin{verbatim}
package com.ran.development.util;

// Импорт классов

public class Utils {

    private Utils() { }
    
    public static <T> Supplier<? extends T> getSupplier(Class<T> parentClass,
            Path classFilePath) {
        try {
            Path folderPath = classFilePath.getParent();
            String className = getClassName(classFilePath);
            ClassLoader loader = new URLClassLoader(new URL[] { folderPath.toUri().toURL() });
            Class<? extends T> cl = (Class<? extends T>)loader.loadClass(className);
            cl.newInstance();
            return () -> {
                try {
                    return cl.newInstance();
                } catch (InstantiationException | IllegalAccessException exception) {
                    String message = "Cannot create instance of " + parentClass.getName() +
                            " subclass";
                    DevelopmentLogging.logger.log(Level.FINE, message, exception);
                    return null;
                }
            };
        } // Перехват исключений, вывод в лог
    }
    
    // Вспомогательные методы
    
}
\end{verbatim}
\chapter{Тестирующий модуль}
\section*{\texttt{TestingSystem.java}}
\begin{verbatim}
package com.ran.testing.system;

public interface TestingSystem {
    
    public void start();
    public void addSubmission(TestingInfo info);
    public void stop();
    
}
\end{verbatim}

\section*{\texttt{MultithreadTestingSystem.java}}
\begin{verbatim}
package com.ran.testing.system;

// Импорт классов

public class MultithreadTestingSystem implements TestingSystem {

    private static final int DEFAULT_THREADS_QUANTITY = Runtime.getRuntime()
            .availableProcessors();
    private static MultithreadTestingSystem defaultSystem = null;
    
    public static MultithreadTestingSystem getDefault() {
        if (defaultSystem == null) {
            defaultSystem = new MultithreadTestingSystem(DEFAULT_THREADS_QUANTITY);
        }
        return defaultSystem;
    }
    
    private int threadsQuantity;
    private TestingFileSupplier fileSupplier;
    private ExecutorService service;
    private ExecutorCompletionService<TestingInfo> completionService;
    private int submissionsCounter = 0;
    private Thread resultHandlersThread;

    // Конструкторы, геттеры, сеттеры

    public synchronized int getSubmissionsCounter() {
        return submissionsCounter;
    }
    
    private synchronized void submissionsCounterUp() {
        submissionsCounter++;
    }
    
    private synchronized void submissionsCounterDown() {
        submissionsCounter--;
    }
    
    @Override
    public void start() {
        service = Executors.newFixedThreadPool(threadsQuantity);
        completionService = new ExecutorCompletionService<>(service);
        resultHandlersThread = new Thread(new ResultTracking());
        resultHandlersThread.start();
    }

    @Override
    public void addSubmission(TestingInfo info) {
        submissionsCounterUp();
        completionService.submit(new TestingTask(info, fileSupplier));
    }

    @Override
    public void stop() {
        service.shutdown();
        resultHandlersThread.interrupt();
    }
    
    private class ResultTracking implements Runnable {
        public void run() {
            boolean interrupted = false;
            while (!interrupted || getSubmissionsCounter() > 0) {
                try {
                    Future<TestingInfo> future = completionService.take();
                    submissionsCounterDown();
                    TestingInfo info = future.get();
                    info.getTestResultHandler().process(info);
                } // Перехват исключений, вывод в лог
            }
        }
    }
    
    private static class TestingTask implements Callable<TestingInfo> {
        private TestingInfo info;
        private TestingFileSupplier fileSupplier;
        // Конструктор
        public TestingInfo call() {
            info.getProblemTester().performTesting(fileSupplier, info);
            return info;
        }
    }
    
}
\end{verbatim}

\section*{\texttt{Verdict.java}}
\begin{verbatim}
package com.ran.testing.system;

public enum Verdict {

    NOT_TESTED,
    WAITING,
    TESTING,
    ACCEPTED,
    PRETESTS_ACC,
    PARTIAL_ACC,
    COMPILE_ERROR,
    RUNTIME_ERROR,
    WRONG_ANSWER,
    TIME_LIMIT,
    MEMORY_LIMIT,
    SECUR_VIOL,
    FAIL

}
\end{verbatim}

\section*{\texttt{VerdictInfo.java}}
\begin{verbatim}
package com.ran.testing.system;

public class VerdictInfo implements Cloneable {

    public static final VerdictInfo VERDICT_NOT_TESTED = new VerdictInfo(Verdict.NOT_TESTED);
    public static final VerdictInfo VERDICT_COMPILE_ERROR = new VerdictInfo(Verdict.COMPILE_ERROR);
    public static final VerdictInfo VERDICT_MEMORY_LIMIT = new VerdictInfo(Verdict.MEMORY_LIMIT);
    public static final VerdictInfo VERDICT_SECUR_VIOL = new VerdictInfo(Verdict.SECUR_VIOL);
    public static final VerdictInfo VERDICT_FAIL = new VerdictInfo(Verdict.FAIL);

    private Verdict verdict;
    private Integer decisionTime;
    private Short decisionMemory;
    private Short points;
    private Integer wrongTestNumber;

    // Конструкторы, геттеры, сеттеры

    @Override
    public VerdictInfo clone() {
        return new VerdictInfo(this.verdict, this.points)
                .setDecisionTime(this.decisionTime)
                .setDecisionMemory(this.decisionMemory)
                .setWrongTestNumber(this.wrongTestNumber);
    }

}
\end{verbatim}

\section*{\texttt{TestingInfo.java}}
\begin{verbatim}
package com.ran.testing.system;

// Импорт классов

public class TestingInfo {

    private TestResultHandler testResultHandler;
    private ProblemTester problemTester;
    private EvaluationSystem evaluationSystem;
    private LanguageToolkit languageToolkit;
    private Checker checker;
    private CodeFileSupplier codeFileSupplier;
    private ProblemFileSupplier problemFileSupplier;
    private boolean pretestsOnly;
    private Integer timeLimit;
    private Short memoryLimit;
    private TestTable testTable;
    private VerdictInfo verdictInfo = null;

    public TestingInfo(TestResultHandler testResultHandler, ProblemTester problemTester,
            EvaluationSystem evaluationSystem, LanguageToolkit languageToolkit,Checker checker,
            CodeFileSupplier codeFileSupplier, ProblemFileSupplier problemFileSupplier,
            boolean pretestsOnly, Integer timeLimit,
            Short memoryLimit, TestTable testTable) {
        this.testResultHandler = testResultHandler;
        this.problemTester = problemTester;
        this.evaluationSystem = evaluationSystem;
        this.languageToolkit = languageToolkit;
        this.checker = checker;
        this.codeFileSupplier = codeFileSupplier;
        this.problemFileSupplier = problemFileSupplier;
        this.pretestsOnly = pretestsOnly;
        this.timeLimit = timeLimit;
        this.memoryLimit = memoryLimit;
        this.testTable = testTable;
    }

    // Геттеры, сеттеры

}
\end{verbatim}

\section*{\texttt{ProblemTester.java}}
\begin{verbatim}
package com.ran.testing.tester;

// Импорт классов

public interface ProblemTester {

    void performTesting(TestingFileSupplier fileSupplier, TestingInfo info);

}
\end{verbatim}

\section*{\texttt{CodingProblemTester.java}}
\begin{verbatim}
package com.ran.testing.tester;

// Импорт классов

public class CodingProblemTester implements ProblemTester {

    @Override
    public void performTesting(TestingFileSupplier fileSupplier, TestingInfo info) {
        Path sourceFile = info.getCodeFileSupplier().getSourceFile();
        Path compileFolder = info.getCodeFileSupplier().getCompileFolder();
        Path configFolder = fileSupplier.getConfigurationFolder();
        int compilationResult = 0;
        try {
            compilationResult = info.getLanguageToolkit().compile(sourceFile,
                    compileFolder, configFolder);
        } catch (FailException exception) {
            TestingLogging.logger.log(Level.FINE,
                    "FailException while compilation of decision", exception);
            if (exception.getCause() != null) {
                TestingLogging.logger.log(Level.FINE,
                        "FailException while compilation of decision (cause)",
                        exception.getCause());
            }
            info.setVerdictInfo(VerdictInfo.VERDICT_FAIL);
            return;
        }
        if (compilationResult != 0) {
            info.setVerdictInfo(VerdictInfo.VERDICT_COMPILE_ERROR);
            return;
        }
        info.getChecker().initialize(info.getProblemFileSupplier());
        EvaluationSystem evaluationSystem = info.getEvaluationSystem();
        evaluationSystem.orderTesting(new CodingTesterDelegate(info, fileSupplier),
                info.isPretestsOnly());
        VerdictInfo verdictInfo = evaluationSystem.getVerdictInfo(info.getTestTable(),
                info.isPretestsOnly());
        info.setVerdictInfo(verdictInfo);
    }
    
    private static class CodingTesterDelegate implements EvaluationSystem.TesterDelegate {

        private TestingInfo info;
        private TestingFileSupplier fileSupplier;
        private Path configFolder;
        // Конструктор
        
        @Override
        public TestTable getTestTable() {
            return info.getTestTable();
        }

        @Override
        public Integer performTestGroup(TestGroupType type, boolean upToFirstFailure) {
            Integer wrongTestNumber = null;
            int testsQuantity = info.getTestTable().getTestsQuantity(type);
            for (int testNumber = 1; testNumber <= testsQuantity; testNumber++) {
                VerdictInfo verdictInfo = performOneTest(type, testNumber);
                info.getTestTable().putVerdictInfo(type, testNumber, verdictInfo);
                if (wrongTestNumber == null && verdictInfo.getVerdict() != Verdict.ACCEPTED) {
                    wrongTestNumber = testNumber;
                }
                if (upToFirstFailure && verdictInfo.getVerdict() != Verdict.ACCEPTED) {
                    break;
                }
            }
            return wrongTestNumber;
        }

        @Override
        public VerdictInfo performOneTest(TestGroupType type, int testNumber) {
            Path outputFile = fileSupplier.getTempFile();
            try {
                if (outputFile == null || Files.notExists(outputFile)) {
                    TestingLogging.logger.fine("Temp file for output was not found");
                    return VerdictInfo.VERDICT_FAIL;
                }
                Path inputFile = info.getProblemFileSupplier()
                        .getTestInputFile(type, testNumber);
                Path answerFile = info.getProblemFileSupplier()
                        .getTestAnswerFile(type, testNumber);
                Path decisionFile = info.getCodeFileSupplier().getCompileFile();
                if (Files.notExists(inputFile) || Files.notExists(answerFile) ||
                        Files.notExists(decisionFile)) {
                    TestingLogging.logger.fine(
                            "Input file or answer file or decision file were not found");
                    return VerdictInfo.VERDICT_FAIL;
                }
                try {
                    LanguageToolkit.ExecutionInfo executionInfo = info.getLanguageToolkit()
                            .execute(decisionFile, inputFile, outputFile, configFolder,
                                    info.getTimeLimit(), info.getMemoryLimit());
                    if (executionInfo.getExitCode() != 0) {
                        return new VerdictInfo(Verdict.RUNTIME_ERROR)
                                .setDecisionTime(executionInfo.getDecisionTime())
                                .setDecisionMemory(executionInfo.getDecisionMemory());
                    }
                    Verdict verdict = info.getChecker().check(inputFile, outputFile, answerFile);
                    return new VerdictInfo(verdict)
                            .setDecisionTime(executionInfo.getDecisionTime())
                            .setDecisionMemory(executionInfo.getDecisionMemory());
                } catch (TimeLimitException exception) {
                    return new VerdictInfo(Verdict.TIME_LIMIT).setDecisionTime(
                            exception.getDecisionTime());
                } catch (MemoryLimitException exception) {
                    return VerdictInfo.VERDICT_MEMORY_LIMIT;
                } catch (SecurityViolatedException exception) {
                    return VerdictInfo.VERDICT_SECUR_VIOL;
                } catch (FailException exception) {
                    TestingLogging.logger.log(Level.FINE,
                            "FailException while execution of decision", exception);
                    if (exception.getCause() != null) {
                        TestingLogging.logger.log(Level.FINE,
                                "FailException while execution of decision (cause)",
                                exception.getCause());
                    }
                    return VerdictInfo.VERDICT_FAIL;
                }
            } finally {
                fileSupplier.deleteTempFile(outputFile);
            }
        }
        
    }

}
\end{verbatim}

\section*{\texttt{EvaluationSystem.java}}
\begin{verbatim}
package com.ran.testing.evaluation;

// Импорт классов

public interface EvaluationSystem {

    void orderTesting(TesterDelegate delegate, boolean pretestsOnly);
    VerdictInfo getVerdictInfo(TestTable testTable, boolean pretestsOnly);
    ProblemResult countProblemResult(TreeMap<Date, VerdictInfo> verdictsMap,
            Date competitionBeginning);

    public interface TesterDelegate {
        TestTable getTestTable();
        Integer performTestGroup(TestGroupType type, boolean upToFirstFailure);
        VerdictInfo performOneTest(TestGroupType type, int testNumber);
    }
    
    public class ProblemResult {
        private short points;
        private int fine;
        // Конструктор, геттеры, сеттеры
    }

}
\end{verbatim}

\section*{\texttt{ICPCEvaluationSystem.java}}
\begin{verbatim}
package com.ran.testing.evaluation;

// Импорт классов

public class ICPCEvaluationSystem implements EvaluationSystem {

    private static final int FINE_FOR_FAILURE = 20;
    
    @Override
    public void orderTesting(TesterDelegate delegate, boolean pretestsOnly) {
        int lastGroupIndex = (pretestsOnly ? TestGroupType.PRETESTS.ordinal() :
                TestGroupType.values().length - 1);
        TestGroupType[] types = Arrays.copyOfRange(TestGroupType.values(), 0, lastGroupIndex);
        for (TestGroupType type: types) {
            Integer wrongTestNumber = delegate.performTestGroup(type, true);
            if (wrongTestNumber != null) {
                break;
            }
        }
    }
    
    @Override
    public VerdictInfo getVerdictInfo(TestTable testTable, boolean pretestsOnly) {
        Integer decisionTime = null;
        Short decisionMemory = null;
        int testsAccepted = 0;
        
        int lastGroupIndex = (pretestsOnly ? TestGroupType.PRETESTS.ordinal() :
                TestGroupType.values().length - 1);
        TestGroupType[] types = Arrays.copyOfRange(TestGroupType.values(), 0, lastGroupIndex);
        
        for (TestGroupType type: types) {
            int testsInGroup = testTable.getTestsQuantity(type);
            for (int testNumber = 1; testNumber <= testsInGroup; testNumber++) {
                VerdictInfo verdictInfo = testTable.getVerdictInfoForTest(type, testNumber);
                if (verdictInfo.getVerdict() != Verdict.ACCEPTED) {
                    return verdictInfo.clone().setWrongTestNumber(testsAccepted + testNumber);
                }
                if (verdictInfo.getDecisionTime() != null) {
                    decisionTime = (decisionTime == null ? verdictInfo.getDecisionTime() :
                            Math.max(decisionTime, verdictInfo.getDecisionTime()));
                }
                if (verdictInfo.getDecisionMemory() != null) {
                    decisionMemory = (decisionMemory == null ? verdictInfo.getDecisionMemory() :
                            (short)Math.max(decisionMemory, verdictInfo.getDecisionMemory()));
                }
            }
            testsAccepted += testsInGroup;
        }
        Verdict verdict = (pretestsOnly ? Verdict.PRETESTS_ACC : Verdict.ACCEPTED);
        return new VerdictInfo(verdict)
                .setDecisionTime(decisionTime)
                .setDecisionMemory(decisionMemory);
    }

    @Override
    public ProblemResult countProblemResult(TreeMap<Date, VerdictInfo> verdictsMap,
            Date competitionBeginning) {
        int failureCounter = 0;
        for (Map.Entry<Date, VerdictInfo> entry: verdictsMap.entrySet()) {
            Verdict verdict = entry.getValue().getVerdict();
            if (verdict != Verdict.ACCEPTED && verdict != Verdict.PRETESTS_ACC) {
                failureCounter++;
            } else {
                int minutesAfterBeginning = (int)((entry.getKey().getTime()
                        - competitionBeginning.getTime()) / 60_000);
                return new ProblemResult((short)1, failureCounter * FINE_FOR_FAILURE +
                        minutesAfterBeginning);
            }
        }
        return new ProblemResult((short)0, 0);
    }

}
\end{verbatim}

\section*{\texttt{LanguageToolkit.java}}
\begin{verbatim}
package com.ran.testing.language;

// Импорт классов

public interface LanguageToolkit {
    
    int compile(Path sourceFile, Path compileFolder, Path configFolder) throws FailException;
    int compile(Path sourceFile, Path compileFolder, Path configFolder,
            OutputStream errorStream) throws FailException;
    ExecutionInfo execute(Path compileFile, Path inputFile, Path outputFile,
            Path configFolder, int timeLimit, short memoryLimit)
            throws FailException, TimeLimitException, MemoryLimitException,
            SecurityViolatedException;

    public class ExecutionInfo {
        private int exitCode;
        private Integer decisionTime;
        private Short decisionMemory;
        // Конструктор, геттеры
    }
    
    static final OutputStream EMPTY_OUTPUT_STREAM = new OutputStream() {
        @Override
        public void write(int b) throws IOException {
        }
    };
    
}
\end{verbatim}

\section*{\texttt{JavaLanguageToolkit.java}}
\begin{verbatim}
package com.ran.testing.language;

// Импорт классов

public class JavaLanguageToolkit implements LanguageToolkit {
    
    private static final String POLICY_FILE_NAME = "java_problem.policy";
    private static final String PROPERTIES_FILE_NAME = "java.properties";
    private static final String JAVA_PATH_PROPERTY = "java_path";
    private static final String STACK_SIZE_PROPERTY = "stack_size";
    
    // Другие статические константы
    
    private static Properties javaProperties = null;
    private static Path javaPropertiesLocation = null;
    
    private static Properties getJavaProperties(Path configFolder) throws FailException {
        if (javaProperties == null || !Objects.equals(configFolder, javaPropertiesLocation)) {
            javaProperties = new Properties();
            try {
                javaProperties.load(Files.newInputStream(
                        configFolder.resolve(PROPERTIES_FILE_NAME)));
            } catch (IOException exception) {
                throw new FailException("Fail because of exception while loading "
                        + PROPERTIES_FILE_NAME, exception);
            }
            javaPropertiesLocation = configFolder;
        }
        return javaProperties;
    }
    
    @Override
    public int compile(Path sourceFile, Path compileFolder, Path configFolder)
            throws FailException {
        return compile(sourceFile, compileFolder, configFolder, EMPTY_OUTPUT_STREAM);
    }

    @Override
    public int compile(Path sourceFile, Path compileFolder, Path configFolder,
            OutputStream errorStream) throws FailException {
        if (Files.notExists(sourceFile) || Files.notExists(compileFolder)) {
            throw new FailException("Compilation failed because files were not found.");
        }
        JavaCompiler compiler = ToolProvider.getSystemJavaCompiler();
        return compiler.run(null, null, errorStream, "-d", compileFolder.toString(),
                sourceFile.toString());
    }

    @Override
    public ExecutionInfo execute(Path compileFile, Path inputFile, Path outputFile,
            Path configFolder, int timeLimit, short memoryLimit)
            throws FailException, TimeLimitException, MemoryLimitException,
            SecurityViolatedException {
        Properties javaProperties = getJavaProperties(configFolder);
        String javaPathProperty = javaProperties.getProperty(JAVA_PATH_PROPERTY);
        String stackSizeProperty = javaProperties.getProperty(STACK_SIZE_PROPERTY);
        Path policyPath = configFolder.resolve(POLICY_FILE_NAME);
        if (Files.notExists(compileFile) || Files.notExists(inputFile) ||
                Files.notExists(outputFile) || Files.notExists(policyPath) ||
                javaPathProperty == null || stackSizeProperty == null) {
            throw new FailException("Execution failed because files or properties were not found.");
        }
        try {
            Path javaPath = Paths.get(javaPathProperty);
            String memoryOption = "-Xmx" + memoryLimit + "M";
            String stackOption = "-Xss" + stackSizeProperty + "M";
            String managerOption = "-Djava.security.manager";
            String policyOption = "-Djava.security.policy==" + policyPath.toString();
            String className = getClassName(compileFile);
            ProcessBuilder processBuilder = new ProcessBuilder("\"" + javaPath.toString() + "\"",
                    memoryOption, stackOption, managerOption, policyOption, className);
            Path compileFolder = compileFile.getParent();
            processBuilder.directory(compileFolder.toFile());
            processBuilder.redirectInput(inputFile.toFile());
            processBuilder.redirectOutput(outputFile.toFile());
            Process process = null;
            try {
                long start = System.currentTimeMillis();
                process = processBuilder.start();
                process.waitFor(timeLimit, TimeUnit.MILLISECONDS);
                int decisionTime = (int)(System.currentTimeMillis() - start);
                if (process.isAlive()) {
                    throw new TimeLimitException(decisionTime);
                }
                int exitValue = process.exitValue();
                if (exitValue != 0) {
                    analyseErrorStream(process.getErrorStream());
                }
                return new ExecutionInfo(exitValue, decisionTime, null);
            } finally {
                if (process != null) {
                    process.destroyForcibly();
                    process.waitFor();
                }
            }
        } catch (InterruptedException | IOException exception) {
            throw new FailException(
                    "InterruptedException or IOException while execution.", exception);
        }
    }
    
    // Вспомогательные методы

}
\end{verbatim}

\section*{\texttt{Checker.java}}
\begin{verbatim}
package com.ran.testing.checker;

// Импорт классов

public interface Checker {

    default void initialize(ProblemFileSupplier problemFileSupplier) { }
    Verdict check(Path inputPath, Path outputPath, Path answerPath);

}
\end{verbatim}

\section*{\texttt{MatchChecker.java}}
\begin{verbatim}
package com.ran.testing.checker;

// Импорт классов

public class MatchChecker implements Checker {
    
    @Override
    public Verdict check(Path inputPath, Path outputPath, Path answerPath) {
        if (Files.notExists(inputPath) || Files.notExists(outputPath)) {
            TestingLogging.logger.fine("Input or output files not found");
            return Verdict.FAIL;
        }
        try (Tokenizer outputTokenizer = new Tokenizer(outputPath);
                Tokenizer answerTokenizer = new Tokenizer(answerPath)) {
            do {
                String outputToken = outputTokenizer.nextToken();
                String answerToken = answerTokenizer.nextToken();
                if (outputToken == null && answerToken == null) {
                    return Verdict.ACCEPTED;
                } else if (!Objects.equals(outputToken, answerToken)) {
                    return Verdict.WRONG_ANSWER;
                }
            } while (true);
        } // Перехват исключений, вывод в лог
    }
    
    private static class Tokenizer implements Closeable {

        private BufferedReader reader;
        private StringTokenizer tokenizer = null;
        
        public Tokenizer(Path path) throws IOException {
            reader = new BufferedReader(new InputStreamReader(Files.newInputStream(
                    path, StandardOpenOption.READ)));
        }
        
        public String nextToken() throws IOException {
            if (tokenizer == null || !tokenizer.hasMoreTokens()) {
                String nextLine = null;
                do {
                    nextLine = reader.readLine();
                } while (Objects.equals(nextLine, ""));
                if (nextLine == null) {
                    return null;
                }
                tokenizer = new StringTokenizer(nextLine);
            }
            return tokenizer.nextToken();
        }
        
        @Override
        public void close() throws IOException {
            reader.close();
        }
        
    }
    
}
\end{verbatim}
\chapter{Модуль для работы с файловой системой}
\section*{\texttt{FileSupplier.java}}
\begin{verbatim}
package com.ran.filesystem.supplier;

// Импорт классов

public interface FileSupplier {

    String addProblemFolder();
    void deleteProblemFolder(String problemFolder);
    List<String> getProblemsFolderNames();
    ProblemDescriptor getProblemDescriptor(String problemFolder);
    Path getProblemStatementPath(String problemFolder);
    boolean putProblemStatementPath(String problemFolder, Path newStatementPath);
    
    boolean addTestInputFiles(String problemFolder, String testGroupType,
            List<Path> inputFilePaths);
    boolean addTestAnswerFile(String problemFolder, String testGroupType, int testNumber,
            Path answerFilePath);
    void deleteTests(String problemFolder, String testGroupType, List<Integer> testNumbers);
    Path getTestInputFile(String problemFolder, String testGroupType, int testNumber);
    Path getTestAnswerFile(String problemFolder, String testGroupType, int testNumber);
    int getTestsQuantity(String problemFolder, String testGroupType);
    TestGroupDescriptor getTestGroupDescriptor(String problemFolder, String testGroupType);
    
    CodeSupplier getCheckerCodeSupplier(String problemFolder);

    String addGeneratorFolder(String problemFolder);
    void deleteGeneratorFolder(String problemFolder, String generatorFolder);
    List<String> getGeneratorFolders(String problemFolder);
    CodeSupplier getGeneratorCodeSupplier(String problemFolder, String generatorFolder);
    
    String addValidatorFolder(String problemFolder);
    void deleteValidatorFolder(String problemFolder, String validatorFolder);
    List<String> getValidatorFolders(String problemFolder);
    CodeSupplier getValidatorCodeSupplier(String problemFolder, String validatorFolder);

    String addAuthorDecisionFolder(String problemFolder);
    void deleteAuthorDecisionFolder(String problemFolder, String authorDecisionFolder);
    List<String> getAuthorDecisionsFolderNames(String problemFolder);
    CodeSupplier getAuthorDecisionCodeSupplier(String problemFolder,
            String authorDecisionFolder);
    AuthorDecisionDescriptor getAuthorDecisionDescriptor(String problemFolder,
            String authorDecisionFolder);

    String addSubmissionFolder();
    void deleteSubmissionFolder(String submissionFolder);
    List<String> getSubmissionsFolderNames();
    SubmissionDescriptor getSubmissionDescriptor(String submissionFolder);
    CodeSupplier getSubmissionCodeSupplier(String submissionFolder);

    Path getTempFile();
    void deleteTempFile(Path path);
    void deleteAllTempFiles();

    Path getConfigurationFolder();
    
}
\end{verbatim}

\section*{\texttt{CodeFileSupplier.java}}
\begin{verbatim}
package com.ran.filesystem.supplier;

import java.nio.file.Path;

public interface CodeSupplier {

    Path getFolder();
    Path getSourceFolder();
    Path getSourceFile();
    Path putSourceFile(Path sourceFile);
    Path getCompileFolder();
    Path getCompileFile();

}
\end{verbatim}

\section*{\texttt{EntityDescriptor.java}}
\begin{verbatim}
package com.ran.filesystem.descriptor;

// Импорт классов

public class EntityDescriptor {

    private Map<String, String> properties = new LinkedHashMap<>();
    private Path path;
    private String rootNodeName;

    public EntityDescriptor(Path path, String rootNodeName) {
        this.path = path;
        this.rootNodeName = rootNodeName;
    }
    
    public String getProperty(String propertyName) {
        return properties.get(propertyName);
    }
    
    public void setProperty(String propertyName, String propertyValue) {
        properties.put(propertyName, propertyValue);
    }
    
    public void load() {
        try {
            DocumentBuilderFactory factory = DocumentBuilderFactory.newInstance();
            DocumentBuilder builder = factory.newDocumentBuilder();
            Document document = builder.parse(path.toFile());
            NodeList nodeList = document.getDocumentElement().getChildNodes();
            for (int i = 0; i < nodeList.getLength(); i++) {
                Node node = nodeList.item(i);
                if (node instanceof Element) {
                    String propertyName = node.getNodeName();
                    String propertyValue = "";
                    if (node.getFirstChild() != null) {
                        propertyValue = node.getFirstChild().getNodeValue();
                    }
                    properties.put(propertyName, propertyValue);
                }
            }
        } catch (Exception exception) {
            FileSystemLogging.logger.log(Level.FINE,
                    "Exception while loading submission descriptor", exception);
        }
    }
    
    public void persist() {
        try (OutputStream stream = Files.newOutputStream(path)) {
            DocumentBuilderFactory factory = DocumentBuilderFactory.newInstance();
            DocumentBuilder builder = factory.newDocumentBuilder();
            Document document = builder.newDocument();
            Node submissionElement = document.createElement(rootNodeName);
            for (Map.Entry<String, String> entry: properties.entrySet()) {
                Node newNode = document.createElement(entry.getKey());
                newNode.appendChild(document.createTextNode(entry.getValue()));
                submissionElement.appendChild(newNode);
            }
            document.appendChild(submissionElement);
            DOMImplementation implementation = document.getImplementation();
            DOMImplementationLS implementationLS = (DOMImplementationLS)implementation
                    .getFeature("LS", "3.0");
            LSSerializer serializer = implementationLS.createLSSerializer();
            serializer.getDomConfig().setParameter("format-pretty-print", true);
            LSOutput output = implementationLS.createLSOutput();
            output.setEncoding("UTF-8");
            output.setByteStream(stream);
            serializer.write(document, output);
        } catch (Exception exception) {
            FileSystemLogging.logger.log(Level.FINE,
                    "Exception while loading submission descriptor", exception);
        }
    }
    
    // Вспомогательные методы
    
}
\end{verbatim}
\chapter{Модуль с графическим интерфейсом}
\section*{\texttt{Main.java}}
\begin{verbatim}
package com.ran.interaction.main;

// Импорт классов

public class Main {

    private static void plugInClasses() {
        JavaClassChecker.plugInClass();
    }

    public static void main(String[] args) {
        plugInClasses();
        MainController controller = new MainController();
        controller.init();
        controller.showFrame();
    }

}
\end{verbatim}

\section*{\texttt{Observer.java}}
\begin{verbatim}
package com.ran.interaction.observer;

@FunctionalInterface
public interface Observer {

    void notify(String id, Object parameter);
    
}
\end{verbatim}

\section*{\texttt{Publisher.java}}
\begin{verbatim}
package com.ran.interaction.observer;

import java.util.List;

public interface Publisher {

    List<String> getAvailableIds();
    void subscribe(String id, Observer observer);
    Observer getObserver(String id);
    
    default void initObservers() {
        for (String id: getAvailableIds()) {
            subscribe(id, EmptyObserver.getInstanse());
        }
    }
    
    default void notifyObservers(Object parameter) {
        for (String id: getAvailableIds()) {
            getObserver(id).notify(id, parameter);
        }
    }
    
}
\end{verbatim}

\section*{\texttt{MainFrame.java}}
\begin{verbatim}
package com.ran.interaction.windows;

// Импорт классов

public class MainFrame extends JFrame {

    private static final String SUBMISSIONS = "Submissions";
    private static final String PROBLEMS = "Problems";
    
    public MainFrame() {
        setDefaultCloseOperation(JFrame.EXIT_ON_CLOSE);
        initComponents();
        initCustomComponents();
    }

    // Код создания интерфейса окна, поля с компонентами                  

    private SubmissionsPanel submissionsPanel;
    private ProblemsPanel problemsPanel;

    private void initCustomComponents() {
        submissionsPanel = new SubmissionsPanel();
        tabbedPane.add(submissionsPanel);
        tabbedPane.setTitleAt(0, SUBMISSIONS);
        problemsPanel = new ProblemsPanel();
        tabbedPane.add(problemsPanel);
        tabbedPane.setTitleAt(1, PROBLEMS);
    }
    
    public SubmissionsPanel getSubmissionsPanel() {
        return submissionsPanel;
    }

    public ProblemsPanel getProblemsPanel() {
        return problemsPanel;
    }
    
}
\end{verbatim}

\section*{\texttt{SubmissionsPanel.java}}
\begin{verbatim}
package com.ran.interaction.panels;

// Импорт классов

public class SubmissionsPanel extends JPanel implements Publisher {

    public static final String ADD = "add_submission";
    public static final String RESUBMIT = "resubmit_submission";
    public static final String DELETE = "delete_submission";
    public static final String UPDATE = "update_submissions";
    public static final String VIEW_CODE = "view_submission_code";
    
    // Дополнительные статические константы
    
    public SubmissionsPanel() {
        initComponents();
        initCustomComponents();
        setTableContent(new Object[][] { });
        initObservers();
    }

    // Код создания интерфейса окна, поля с компонентами                     

    private Map<String, Observer> observers = new HashMap<>();
    
    private void initCustomComponents() {
        buttonAdd.addActionListener(event -> getObserver(ADD).notify(ADD, null));
        buttonResubmit.addActionListener(event -> callObserverIfRowIsSelected(RESUBMIT));
        buttonDelete.addActionListener(event -> callObserverIfRowIsSelected(DELETE));
        buttonViewCode.addActionListener(event -> callObserverIfRowIsSelected(VIEW_CODE));
        buttonUpdate.addActionListener(event -> getObserver(UPDATE).notify(UPDATE, null));
    }
    
    private void callObserverIfRowIsSelected(String id) {
        if (tableSubmissions.getSelectedIdentifier() != null) {
            getObserver(id).notify(id, tableSubmissions.getSelectedIdentifier());
        }
    }
    
    @Override
    public List<String> getAvailableIds() {
        return Arrays.asList(ADD, RESUBMIT, DELETE, VIEW_CODE, UPDATE);
    }
    
    @Override
    public void subscribe(String id, Observer observer) {
        observers.put(id, observer);
    }

    @Override
    public Observer getObserver(String id) {
        return observers.getOrDefault(id, EmptyObserver.getInstanse());
    }
    
    public final void setTableContent(Object[][] content) {
        tableSubmissions.setTableContent(content, TABLE_HEADERS);
    }
    
}
\end{verbatim}

\section*{\texttt{ProblemsPanel.java}}
\begin{verbatim}
package com.ran.interaction.panels;

// Импорт классов

public class ProblemsPanel extends JPanel implements Publisher {
    
    public static final String ADD = "add_problem";
    public static final String EDIT = "edit_problem";
    public static final String DELETE = "delete_problem";
    public static final String UPDATE = "update_problems";
    
    // Дополнительные статические константы
    
    public ProblemsPanel() {
        initComponents();
        initCustomComponents();
        setTableContent(new Object[][] { });
        initObservers();
    }

    // Код создания интерфейса окна, поля с компонентами

    private Map<String, Observer> observers = new HashMap<>();
    
    private void initCustomComponents() {
        buttonAdd.addActionListener(event -> getObserver(ADD).notify(ADD, null));
        buttonEdit.addActionListener(event -> callObserverIfRowIsSelected(EDIT));
        buttonDelete.addActionListener(event -> callObserverIfRowIsSelected(DELETE));
        buttonUpdate.addActionListener(event -> getObserver(UPDATE).notify(UPDATE, null));
    }
    
    private void callObserverIfRowIsSelected(String id) {
        if (tableProblems.getSelectedIdentifier() != null) {
            getObserver(id).notify(id, tableProblems.getSelectedIdentifier());
        }
    }
    
    @Override
    public List<String> getAvailableIds() {
        return Arrays.asList(ADD, EDIT, DELETE, UPDATE);
    }
    
    @Override
    public void subscribe(String id, Observer observer) {
        observers.put(id, observer);
    }

    @Override
    public Observer getObserver(String id) {
        return observers.getOrDefault(id, EmptyObserver.getInstanse());
    }
    
    public final void setTableContent(Object[][] content) {
        tableProblems.setTableContent(content, TABLE_HEADERS);
    }
    
}
\end{verbatim}

\section*{\texttt{MainController.java}}
\begin{verbatim}
package com.ran.interaction.controllers;

// Импорт классов

public class MainController {

    private ProblemsCreator creator;
    private MainFrame mainFrame;

    public void init() {
        creator = new ProblemsCreator();
        creator.init();
    }
    
    public void showFrame() {
        EventQueue.invokeLater(() -> {
            try {
                for (UIManager.LookAndFeelInfo info : UIManager.getInstalledLookAndFeels()) {
                    if ("Nimbus".equals(info.getName())) {
                        UIManager.setLookAndFeel(info.getClassName());
                        break;
                    }
                }
            } catch (ClassNotFoundException | InstantiationException |
                    IllegalAccessException | UnsupportedLookAndFeelException exception) {
                InteractionLogging.logger.log(Level.FINE,
                        "Cannot set Nimbus Look and Feel", exception);
            }
            mainFrame = new MainFrame();
            configurateMainFrame(mainFrame);
            mainFrame.setVisible(true);
        });
    }

    private void configurateMainFrame(MainFrame mainFrame) {
        mainFrame.addWindowListener(new WindowAdapter() {
            public void windowClosing(WindowEvent event) {
                creator.stop();
            }
        });
        SubmissionsPanel submissionsPanel = mainFrame.getSubmissionsPanel();
        updateSubmissions(null, null);
        submissionsPanel.subscribe(SubmissionsPanel.ADD, this::addSubmission);
        submissionsPanel.subscribe(SubmissionsPanel.RESUBMIT, this::submitSubmission);
        submissionsPanel.subscribe(SubmissionsPanel.VIEW_CODE, this::viewSubmissionCode);
        submissionsPanel.subscribe(SubmissionsPanel.DELETE, this::deleteSubmission);
        submissionsPanel.subscribe(SubmissionsPanel.UPDATE, this::updateSubmissions);
        ProblemsPanel problemsPanel = mainFrame.getProblemsPanel();
        updateProblems(null, null);
        problemsPanel.subscribe(ProblemsPanel.ADD, this::addProblem);
        problemsPanel.subscribe(ProblemsPanel.EDIT, this::editProblem);
        problemsPanel.subscribe(ProblemsPanel.DELETE, this::deleteProblem);
        problemsPanel.subscribe(ProblemsPanel.UPDATE, this::updateProblems);
    }
    
    private void addSubmission(String id, Object parameter) {
        SubmissionCreationController creationController = new SubmissionCreationController();
        creationController.setFileSupplier(creator.getFileSupplier());
        creationController.showDialog();
        String submissionFolder = creationController.getSubmissionFolder();
        if (submissionFolder != null) {
            updateSubmissions(null, null);
            submitSubmission(null, submissionFolder);
        }
    }
    
    private void submitSubmission(String id, Object parameter) {
        String submissionFolder = parameter.toString();
        SubmissionResultController resultController = new SubmissionResultController();
        resultController.setFileSupplier(creator.getFileSupplier());
        resultController.setTestingSystem(creator.getTestingSystem());
        resultController.setSubmissionFolder(submissionFolder);
        resultController.showDialog();
        updateSubmissions(null, null);
    }
    
    private void viewSubmissionCode(String id, Object parameter) {
        String submissionFolder = parameter.toString();
        Path sourceFilePath = creator.getFileSupplier()
                .getSubmissionCodeSupplier(submissionFolder).getSourceFile();
        FileEditorController editorController = new FileEditorController();
        editorController.showDialog(sourceFilePath, true);
    }
    
    private void deleteSubmission(String id, Object parameter) {
        int answer = SwingUtil.showYesNoDialog(mainFrame,
                SubmissionsPanel.DELETING_MESSAGE, SubmissionsPanel.DELETING_TITLE);
        if (answer == JOptionPane.YES_OPTION) {
            creator.getFileSupplier().deleteSubmissionFolder(parameter.toString());
            updateSubmissions(null, null);
        }
    }
    
    private void updateSubmissions(String id, Object parameter) {
        FileSupplier fileSupplier = creator.getFileSupplier();
        Properties properties = PresentationSupport.getPresentationProperties();
        List<String> submissionNumbers = fileSupplier.getSubmissionsFolderNames();
        Object[][] content = SwingUtil.prepareTableContent(submissionNumbers, (number, row) -> {
            row.add(number);
            SubmissionDescriptor descriptor = fileSupplier.getSubmissionDescriptor(number);
            String problemNumber = descriptor.getProblemName();
            row.add(fileSupplier.getProblemsFolderNames().contains(problemNumber) ?
                    fileSupplier.getProblemDescriptor(problemNumber).getProblemName() : "");
            row.add(properties.getProperty(descriptor.getEvaluationType()));
            row.add(properties.getProperty(descriptor.getCompilatorName()));
            row.add(TestingUtil.getVerdictDescription(descriptor.getVerdict(),
                    descriptor.getDecisionPoints(), descriptor.getWrongTestNumber()));
            row.add(TestingUtil.getTimeDescription(descriptor.getDecisionTime()));
        });
        mainFrame.getSubmissionsPanel().setTableContent(content);
    }
    
    private void addProblem(String id, Object parameter) {
        creator.getFileSupplier().addProblemFolder();
        updateProblems(null, null);
    }
    
    private void editProblem(String id, Object parameter) {
        ProblemController problemController = new ProblemController();
        problemController.setFileSupplier(creator.getFileSupplier());
        problemController.setTestingSystem(creator.getTestingSystem());
        problemController.setProblemFolder(parameter.toString());
        problemController.showDialog();
        updateProblems(null, null);
        updateSubmissions(null, null);
    }
    
    private void deleteProblem(String id, Object parameter) {
        int answer = SwingUtil.showYesNoDialog(mainFrame,
                ProblemsPanel.DELETING_MESSAGE, ProblemsPanel.DELETING_TITLE);
        if (answer == JOptionPane.YES_OPTION) {
            creator.getFileSupplier().deleteProblemFolder(parameter.toString());
            updateProblems(null, null);
        }
    }
    
    private void updateProblems(String id, Object parameter) {
        FileSupplier fileSupplier = creator.getFileSupplier();
        List<String> problemNumbers = fileSupplier.getProblemsFolderNames();
        Properties presentationProperties = PresentationSupport.getPresentationProperties();
        Object[][] content = SwingUtil.prepareTableContent(problemNumbers, (number, row) -> {
            row.add(number);
            ProblemDescriptor descriptor = fileSupplier.getProblemDescriptor(number);
            row.add(descriptor.getProblemName());
            row.add(TestingUtil.getTimeDescription(descriptor.getTimeLimit()));
            row.add(TestingUtil.getMemoryDescription(descriptor.getMemoryLimit()));
            row.add(presentationProperties.getProperty(descriptor.getCheckerType()));
        });
        mainFrame.getProblemsPanel().setTableContent(content);
    }

}
\end{verbatim}

\section*{\texttt{ProblemsCreator.java}}
\begin{verbatim}
package com.ran.interaction.controllers;

// Импорт классов

public class ProblemsCreator {

    private static final String TESTING_SYSTEM_THREADS_DEFAULT = "10";
    private static final String FILE_SYSTEM_PATH_DEFAULT = System.getProperty("user.dir");

    private TestingSystem testingSystem;
    private FileSupplier fileSupplier;

    public void init() {
        Map<String, String> creatorParams = new CreatorParamsReader().getCreatorParams();
        String fileSystemPath = creatorParams.getOrDefault(
                CreatorParamsReader.FILE_SYSTEM_PATH_PARAM_NAME, FILE_SYSTEM_PATH_DEFAULT);
        fileSupplier = new StandardFileSupplier(Paths.get(fileSystemPath));
        String testingSystemThreads = creatorParams.getOrDefault(CreatorParamsReader
                .TESTING_SYSTEM_THREADS_PARAM_NAME, TESTING_SYSTEM_THREADS_DEFAULT);
        MultithreadTestingSystem multithreadTestingSystem
                = MultithreadTestingSystem.getDefault();
        multithreadTestingSystem.setThreadsQuantity(Integer.parseInt(testingSystemThreads));
        multithreadTestingSystem.setFileSupplier(new TestingFileSupplierImpl(fileSupplier));
        testingSystem = multithreadTestingSystem;
        testingSystem.start();
    }

    public void stop() {
        testingSystem.stop();
    }

    public TestingSystem getTestingSystem() {
        return testingSystem;
    }

    public FileSupplier getFileSupplier() {
        return fileSupplier;
    }

}

class CreatorParamsReader {

    public static final String CREATOR_PARAMS_FILE = "creator-params.xml";
    public static final String TESTING_SYSTEM_THREADS_PARAM_NAME = "testingSystemThreads";
    public static final String FILE_SYSTEM_PATH_PARAM_NAME = "fileSystemPath";
    public static final String PARAM_NAME_NODE_NAME = "name";
    public static final String PARAM_VALUE_NODE_NAME = "value";

    public Map<String, String> getCreatorParams() {
        Map<String, String> creatorParams = new HashMap<>();
        try (InputStream creatorParamStream = ProblemsCreator.class
                .getResourceAsStream(CREATOR_PARAMS_FILE)) {
            DocumentBuilder documentBuilder = DocumentBuilderFactory.newInstance()
                    .newDocumentBuilder();
            Document document = documentBuilder.parse(creatorParamStream);
            Element rootElement = document.getDocumentElement();
            NodeList paramList = rootElement.getChildNodes();
            for (int i = 0; i < paramList.getLength(); i++) {
                Node node = paramList.item(i);
                if (node instanceof Element) {
                    updateCreatorParamsMap(creatorParams, node);
                }
            }
        } catch (Exception exception) {
            InteractionLogging.logger.log(Level.FINE,
                    "Exception while reading creator-params.xml", exception);
        }
        return creatorParams;
    }

    private void updateCreatorParamsMap(Map<String, String> creatorParams, Node paramNode) {
        NodeList nodeList = paramNode.getChildNodes();
        String paramName = null;
        String paramValue = null;
        for (int i = 0; i < nodeList.getLength(); i++) {
            Node node = nodeList.item(i);
            if (node instanceof Element) {
                String nodeName = node.getNodeName();
                String nodeValue = node.getFirstChild().getNodeValue();
                switch (nodeName) {
                    case PARAM_NAME_NODE_NAME: paramName = nodeValue; break;
                    case PARAM_VALUE_NODE_NAME: paramValue = nodeValue; break;
                }
            }
        }
        creatorParams.put(paramName, paramValue);
    }
}
\end{verbatim}

\end{document}
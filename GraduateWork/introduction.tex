Олимпиада по программированию "--- это интеллектуальное соревнование по решению различных задач на ЭВМ, для решения которых необходимо придумать и применить какой-либо алгоритм и программу на одном из языков программирования \cite{wiki}. Проводятся такие олимпиады сегодня как среди школьников, так и среди студентов и профессиональных работников, как на уровне вузов, так и на международном уровне. Многие крупные IT-компании, такие как Google, Facebook, Яндекс, Mail.ru, ВКонтакте, регулярно проводят свои собственные олимпиады, что, несомненно, говорит об их актуальности на сегодняшний день.

На олимпиаде по программированию участникам предлагается набор из нескольких задач, решением каждой задачи является программа, написанная на одном из разрешённых языков программирования. Программа должна считывать данные указанного формата из некоторого входного потока, обрабатывать их согласно условию задачи и выводить ответ в выходной поток. Чтобы решение было засчитано как верное, необходимо, чтобы оно выводило правильные ответы на заранее определённом наборе тестов.

Проверка решений на тестах производится с помощью специальных тестирующих систем. Такая система должна уметь компилировать код решений с помощью различных компиляторов, запускать решения на определённых наборах тестов, проверять корректность ответов простым сравнением с эталонными выходными данными или более сложными способами (например, с помощью специальных чекеров), а также "--- выносить вердикты по каждому полученному решению. Кроме того, система должна поддерживать различные системы оценивания решений (такие как IOI и ICPC \cite{wiki}). Наконец, от тестирующей системы требуется, чтобы она проверяла решения как можно быстрее, для чего, например, проверка может выполняться параллельно в нескольких потоках. Существует несколько известных тестирующих систем, используемых в реальных олимпиадах, например: Ejudge, PCMS2, Contester, Testsys.

Разработка задач для олимпиад по программированию "--- отдельный процесс, который также является весьма трудоёмким. В него входит написание корректного и содержательного условия и подготовка большого набора тестов, которые должны охватывать все возможные частные случаи в задаче. Тесты могут представлять собой данные очень больших объёмов, поэтому для их подготовки также необходимы специальное программное обеспечение и инструменты, такие как генераторы, валидаторы, чекеры и интеракторы, у каждого из которых есть своё особое предназначение. В качестве примера известной системы разработки задач можно привести систему Polygon \cite{polygon}, на которой подготавливают задачи для проведения контестов на платформе Codeforces \cite{codeforces}.

В данной работе мы поставим перед собой цель изучить основные аспекты работы тестирующих систем и систем разработки задач для олимпиад по программированию, а также "--- написать программу с удобным графическим пользовательским интерфейсом, предоставляющую возможность разрабатывать новые задачи и проверять решения на подготовленных тестах. Предполагается написание программы на языке Java с использованием библиотеки Swing.